\begin{resumen}
% Resumen: IMRD, 250 palabras

% Introducción al problema
%  - Cuál es?
%  - Por qué es importante resolverlo?
% Cómo puede ser o es resuelto?
%  - Qué problemas tiene este proceso?
% Propuesta para mejorar el proceso de resolución del problema
%  - Modelo
%  - Resultados  

El estudio de la argumentación en la prensa cubana es un campo se han reportado
relativamente pocas investigaciones. En estos estudios es posible obtener 
información de los esquemas argumentativos utilizados en los textos y
realizar acciones en base a estos.
Este problema tradicionalmente es resuelto mediante 
una anotación manual por expertos en lingüística, trabajo que se 
caracteriza por tomar mucho tiempo y recursos. La Extracción
de Argumentos es la rama del Procesamiento del Lenguaje Natural
encargada de estudiar algoritmos y métodos para solucionar los problemas
asociados a la anotación de las estructuras argumentativas. Mediante el uso 
de estos algoritmos es posible automatizar el procedimiento de anotación
de la argumentación. 
En este trabajo se propone la anotación de textos argumentativos
mediante el uso de dos modelos de aprendizaje profundo, entrenados con 
conjuntos de datos traducidos y proyectados del inglés, encargados de resolver
las tareas relacionadas al problema. 
El primer modelo propuesto 
consiste en uno secuencia a secuencia usado para la extracción y clasificación
de las unidades de discurso argumentativas (UDA) mediante el uso de \emph{Long Short Term Memory} 
(LSTM) y \emph{Conditional Random Field} (CRF). Para la extracción y clasificación de 
enlaces entre las UDAs se propone un modelo de clasificación basado en redes residuales,
atención y LSTM. Ambos modelos utilizan \emph{embeddings} GloVe para la representación 
de las palabras. Los resultados obtenidos en la extracción de UDAs alcanzaron valores de
0,82 en la métrica F1 comparados con 0,85 obtenidos en el estado del arte. 
En las demás tareas, los resultados no son directamente comparables con los del estado del arte, 
los mejores valores F1 obtenidos fueron 0,56 en la clasificación de UDAs, 0,74 en la predicción
de enlaces y 0,39 en la clasificación de enlaces.
Con dichos modelos se anotaron las ``Cartas a la Dirección'', del 
periódico Granma, creándose un conjunto de datos con las estructuras argumentativas anotadas
y listas para el estudio de estas por lingüistas.

\end{resumen}

\begin{abstract}
	% Resumen en inglés

The study of argumentation in the Cuban press is a field in which relatively little
research has been reported. In these studies it is possible to obtain information on 
the argumentative schemes used in the texts and take actions based on them. This problem
is traditionally solved through manual annotation by linguistic experts, a work that takes 
a lot of time and resources. Argument Extraction is the branch of Natural Language Processing 
in charge of studying algorithms and methods to solve the problems associated with the annotation 
of argument structures. By using these algorithms it is possible to automate the argumentation 
annotation procedure.  In this paper we propose the annotation of argumentative texts by using 
two deep learning models, trained with translated and projected English datasets, in charge of 
solving the tasks related to the problem.  The first proposed model consists of a sequence to 
sequence one used for the extraction and classification of argumentative discourse units (ADUs) 
by using Long Short Term Memory (LSTM) and Conditional Random Field (CRF). A classification model 
based on residual networks, attention and LSTM is proposed for the extraction and classification of 
links between ADUs. Both models use GloVe for word representation. The results obtained in the 
extraction of ADUs reached values of 0.82 in the F1 metric compared to 0.85 obtained in the state 
of the art.  In the other tasks, the results are not directly comparable with those of the state of 
the art, the best F1 values obtained were 0.56 in UDAs classification, 0.74 in link prediction and 
0.39 in link classification. With these models, the ``Letters to the Management'' of the Granma 
newspaper were annotated, creating a data set with the argumentative structures annotated and 
ready to be studied by linguists. 

\end{abstract}