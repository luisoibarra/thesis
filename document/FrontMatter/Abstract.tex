\begin{resumen}
% Resumen: IMRD, 250 palabras

% Introducción al problema
%  - Cuál es?
%  - Por qué es importante resolverlo?
% Cómo puede ser o es resuelto?
%  - Qué problemas tiene este proceso?
% Propuesta para mejorar el proceso de resolución del problema
%  - Modelo
%  - Resultados  

% TODO Mejorar el inicio

El estudio de la argumentación en la prensa cubana es un campo el cual no se 
han realizado investigaciones. Con estos datos es posible obtener 
información de los esquemas argumentativos utilizados en los textos y
realizar acciones en base a estos.
Este problema tradicionalmente es resuelto mediante 
una anotación manual por expertos en lingüística, trabajo que se 
caracteriza por tomar una gran cantidad de tiempo y recursos. La Extracción
de Argumentos (EA) es la rama del Procesamiento del Lenguaje Natural (PLN) 
encargada de estudiar algoritmos y métodos para solucionar los problemas
asociados a la anotación de las estructuras argumentativas. Mediante el uso 
de estos algoritmos es posible automatizar el procedimiento de anotación
de la argumentación. El primer modelo propuesto 
consiste en uno secuencia a secuencia usado para la extracción y clasificación
de las unidades de discurso argumentativas (UDA) mediante el uso de LSTM y 
CRF. Para la extracción y clasificación de 
enlaces entre las UDAs se propone un modelo de clasificación basado en redes residuales,
atención y LSTM. Ambos modelos utilizan \emph{embeddings} GloVe para la representación 
de las palabras. Los resultados obtenidos alcanzaron valores similares en comparación
con el estado del arte en las tareas de extracción de las UDAs, los resultados obtenidos 
en las demás tareas no son directamente comparables con los estados del arte, aunque en 
la tarea de extracción de enlace se alcanzaron buenos resultados.
Con dichos modelos se anotaron las Cartas a la Dirección, sección del 
periódico Granma, creándose un conjunto de datos con las estructuras argumentativas anotadas
y listas para el estudio de estas.

% Este campo 
% se ha ido desarrollando en los últimos años aunque en español no se encuentran 
% conjuntos de datos con los cual entrena

% La Extracción de Argumentos (EA) permite la extracción y clasificación de componentes argumentativas 
% y sus relaciones en textos. El desarrollo de esta ha ido en aumento en los últimos años, aunque la 
% presencia de su estudio en el lenguaje español ha sido baja.  

% Dicha rama del Procesamiento de Lenguaje Natural tiene pocos 
% corpus disponibles en español, por lo que se plantea el uso de técnicas de traducción automática 
% y de proyección de anotaciones del inglés al español para la construcción de corpus en este 
% lenguaje \cite{eger2018cross} . En el trabajo se propone un modelo enfocado en el uso de 
% Redes Neuronales Artificiales con modelos de atención para el análisis de 
% textos de prensa cubana, específicamente la versión digital del periódico Granma.

% Palabras clave: Projección de etiquetas, Extracción de Argumentos, 

% (TODO Poner resultados)

\end{resumen}

\begin{abstract}
	% Resumen en inglés

The study of argumentation in the Cuban press is a field in which no research has been 
carried out. With this data it is possible to obtain information on the argumentative 
schemes used in the texts and to carry out actions based on them. This problem is traditionally 
solved through manual annotation by linguistic experts, a task that takes a great deal of time 
and resources. Argument Extraction (AE) is the branch of Natural Language Processing (NLP) 
in charge of studying algorithms and methods to solve the problems associated with the 
annotation of argument structures. By using these algorithms it is possible to automate the 
argumentation annotation procedure. The first proposed model consists of a sequence to sequence 
model used for the extraction and classification of argumentative discourse units (ADUs) using 
LSTM and CRF. For the extraction and classification of links between ADUs, a classification model 
based on residual networks, attention and LSTM is proposed. Both models use embeddings GloVe for 
word representation. The results obtained reached similar values compared to the state of the art 
in the UDAs extraction tasks, the results obtained in the other tasks are not directly comparable
with the state of the art, although in the link extraction task good results were achieved. With 
these models, the Letters to the Director, section of the Granma newspaper, were annotated, creating 
a data set with the argumentative structures annotated and ready for their study.

\end{abstract}