\begin{resumen}

% La Minería de Argumentos permite la extracción y clasificación de componentes argumentativas 
% y sus relaciones en textos. Dicha rama del Procesamiento de Lenguaje Natural tiene pocos 
% corpus disponibles en español, por lo que se plantea el uso de técnicas de traducción automática 
% y de proyección de anotaciones del inglés al español para la construcción de corpus en este 
% lenguaje \cite{eger2018cross} . En el trabajo se propone un modelo enfocado en el uso de 
% Redes Neuronales Artificiales con modelos de atención para el análisis de 
% textos de prensa cubana, específicamente la versión digital del periódico Granma.

% Palabras clave: Projección de etiquetas, Extracción de Argumentos, 

% (TODO Poner resultados)

\end{resumen}

\begin{abstract}
	Resumen en inglés
\end{abstract}