\begin{conclusions}

% TODO Como se cumplieron los objetivos de la introduccion y para que sirve lo que hice
% lo que hice para que puede servir, 
% en que escenarios, 
% es competitivo con el estado del arte, 
% por qué sí o por qué no
    
En el trabajo se logró la extracción de estructuras argumentativas en los textos de las 
Cartas a la Dirección del periódico Granma. Para esto se 
extrajeron los textos de las Cartas creando un conjunto de datos
% TODO PREGUNTAR dice cambiar {si no,} por {sis}
conteniendo, no solo el texto de las cartas, si no, los comentarios 
escritos por los usuarios e información sobre la carta a la que responde, si es una respuesta.
Se creó un software que puede ser utilizado no solo para el trabajo con las Cartas a la Dirección, si no
que este permite un uso general para las tareas de proyección de corpus y trabajo relacionados a la EA.
Este permite incorporar nuevas componentes haciendo posible una extensión simple y desacomplada. 
Para la anotación de las estructuras se crearon conjuntos de datos en español a partir de conjuntos 
anotados en inglés para poder 
tener datos con los cuales entrenar los modelos propuestos. Estos modelos uno era el encargado 
de la segmentación y clasificación de las UDAs, y otro de la extracción y clasificación de las 
relaciones entre estas, los cuales fueron utilizados para la anotación de ls textos extraídos.
Los resultados obtenidos en la tarea de segmentación se encuentran al nivel del estado del arte,
en las demás tareas no se encontraron comparaciones directas al enfoque tomado en la propuesta,
aunque no teniendo en cuenta esto, se obtienen comparativas inferiores en la clasificación
de UDAs y en la clasificación de relaciones, aunque se supera la métrica de predicción de enlace. 
Los resultados obtenidos del procesamiento de las Cartas a la Dirección son considerados 
aceptables por autor. 

\end{conclusions}
