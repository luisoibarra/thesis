\begin{recomendations}

    La principal dificultad en el trabajo fue la carencia de un conjunto de anotados
    sobre el tema en específico relacionado con la extracción de argumentos en la prensa.
    Por lo que se propone la creación de este conjunto para poder realizar una mejor 
    validación y entrenamiento del modelo propuesto. También se considera la creación de un servicio 
    online basado en Brat para la socialización y mejora de los resultados obtenidos.
    El uso de representaciones BERT ha llevado a muchas tareas de PLN a nuevos estados 
    del arte, por lo tanto se propone investigar el uso de estos \emph{embeddings} en 
    el modelo. El problema principal obtenido en el modelo fue relacionado con la 
    predicción de enlaces, un problema que tiene el modelo es la falta de contexto global
    del texto para hacer la predicción, por lo que se insta a la búsqueda y experimentación
    de métodos que tomen esto en cuenta, una variante son las \emph{Graph Neural Networks}.

    % \begin{enumerate}
    %     \item Usar metodos de Graph Neural Networks para modelar las relaciones entre las UDA. (Actualmente las relaciones extraídas son independientes)
    %     \item Ajustar (Tune) awesome-align con inglés-español.
    %     \item Usar otros embeddings para la representación de palabras (BERT, SciBERT, RoBERTa).
    %     \item Desplegar un servidor Brat para socializar los resultados y permitir que los investigadores perfeccionen las anotaciones.
    %     \item Validación externa con lingüistas que permitan generalizar el uso del modelo.
    % \end{enumerate}
\end{recomendations}
