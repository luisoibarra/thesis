\chapter{Estado del Arte}\label{chapter:state-of-the-art}

% \textbf{Puntos claves}
% \begin{itemize}
%     \item Ver los marcos teóricos de la argumentación existentes. (Freeman, Toulmin, Standard?)
%     \item Objetivos de la extracción de argumentos (Predecir las estructras argumentativas de un texto plano).
%     \item Modelos presentados para la solución de estos problemas. (End-to-end, Pipeline, mixed) (Problema de clasificación, seq2seq, MTL, link prediction)
%     \item Cómo se han evaluado los modelos. (F1, Precision, Recall, 50\% F1, 100\% F1) 
% \end{itemize}

% Recomendaciones sobre el contenido
% \begin{itemize}
%     \item Tener referencias a trabajos relevantes
%     \item Presentar marco teórico
%     \item Fundamentación teórica
%     \item Herramientas a utilizar
%     \begin{itemize}
%         \item Justificar el uso de la herramienta, poner ejemplos comparativos con otras herramientas
%     \end{itemize}
% \end{itemize}

% Una explicación del problema (metáfora), cómo han resuelto el problema y qué métricas, qué valores, 
% moralejas (Enseñanzas que dejaron los papers leídos) (5 páginas)

\section{Argumentación}

La argumentación es una actividad en donde el ponente trata de justificar racionalmente 
sus puntos de vista sobre ciertos temas, con el objetivo de persuadir al receptor a aceptar
las afirmaciones emitidas.  
Con el tiempo se ha podido elaborar una teoría en la cual basarse para el estudio de esta. 

\subsection{Argumentación Clásica}

Las primeras citas de la argumentación provienen de la retórica encontradas en \emph{On Rhetoric} 
escrito por Aristóteles, lo cual se puede considerar como la argumentación clásica. 
En este el autor identifica tipos de persuación, a través de la credibilidad, lógica, 
emoción, disposición del receptor o receptores al contenido.
Además se introduce los componentes que presentan la argumentación clásica, tales como
introducción, contexto y antecedentes, posición y argumentos del ponente, pruebas negativas o positivas
que soporten los argumentos dados y conclusión y llamado a acción. [\cite{classicalArgument}] 

\subsection{Argumentación Rogeriana}

Este es visto en debates, en los cuales existen varios puntos de vista y se desea llegar a una 
decisión mutualmente beneficiosa para todas las partes. Desarrollado por Carl Rogers, es basado en 
la comprensión de los puntos de vista del oponenete teniendo en cuenta tres principios 
[\cite{rogerianArgument}]:

\begin{enumerate}
    \item Escuchar y hacer entender al oponente que es entendido.
    \item Encontrar validez en la posición del oponente.
    \item Inducir al oponente que sus deseos no difieren de lo dicho por el ponente.
\end{enumerate}

\subsection{Método Toulmin}

El Método Toulmin fue extrapolado del libro \emph{The Uses of Argument}[\cite{toulmin_2003}] escrito por Stephen E. Toulmin.
Este método divide los argumentos en seis partes: afirmación (claim), fundamento (grounds), 
justificación (warrant), calificador (qualifier), refutación (rebuttal) y respaldo (backing).
Mediante las afirmaciones se conoce el argumento principal que el autor quiere probar a la audiencia,
estas son respaldados con fundamentos siendo estos las evidencias y hechos en que se apoya el autor.
Las justificaciones pueden estar explícitas o implícitas y son suposiciones que vinculan los
fundamentos con las afirmaciones, estas a su vez pueden ser respaldadas por conocimiento.
El esquema introduce la posibilidad de otra sitaución válida a la establecida en las afirmaciones
mediante la refutación. Los calificadores son usados para dar más información de la calidad o seguridad
de las afirmaciones dadas. Un ejemplo de este esquema es:

\emph{[Se escucharon ladridos y aullidos en la distancia](fundamento), [probablemente](calificador) 
[haya perros en las cercanías](afirmación).}

En este ejemplo, además de las partes explícitas, se encuentran implícitas, la justificación 
(los perros son animales que ladran y aullan), el respaldo (se sabe que existen perros en la zona) y 
la refutación (Puede ser que hayan lobos o coyotes cerca). [\cite{toulminArgument}]

\subsection{Conclusiones}

El método Toulmin presenta una popular y definida forma de representar las estructuras de la argumentación, 
además de haber sido estudiado y ampliado [\cite{freeman2011argument}] y usado en diferentes 
investigaciones [\cite{stab2017parsing}, \cite{niculae2017argument}]. Este constituye una base
teórica simple y rica para poder ser entendida con facilidad y expresar las diferentes partes de la 
argumentación. 

\section{Extracción de Argumentos}

% PONER LAS METRICAS? PERO LOS CORPUS A VECES SON DIFERENTES
% LUEGO EXPONER LAS DIFICULTADES: Pipeline, End-to-end, Features manuales, Features de gaceteras (goudas2015)

La Extracción de Argumentos constituye en la tarea de extracción de las estructuras argumentativas
de texto plano. Existen varias maneras de representar estas estructuras. Entre estas se encuentra
el Argument Interchange Format (AIF) [\cite{rahwan2009aif}], este es un formato en el cual la argumentación
es representada como un grafo con diferentes tipos de nodos, las relaciones entre estos son modelados con 
nodos igualmente para evitar la necesidad de clasificar aristas. También se encuentra Inference Anchoring Theory 
(IAT) [\cite{corneli2019iat}] el cual presenta una versión similar a AIF con una visión orientada al diálogo.
En general se pueden observar como un grafo dirigido en el cual los nodos representan las unidades de
discurso argumentativas (UDA) y las aristas representan las relaciones entre estas, tanto los nodos como
las aristas se clasifican en dependencia del modelo argumentativo usado.
El problema anterior presenta una serie de elementos que son necesarios resolver:

\begin{itemize}
    \item Extracción de las UDAs del texto.
    \item Clasificación de las UDAs.
    \item Extracción de las relaciones entre las UDAs.
    \item Clasificación de las relaciones extraídas.
\end{itemize}

Una UDA constituye en una sección continua de texto que contiene lo que es considerado información
relevante para el proceso argumentativo. 

La extracción de UDAs consiste en separar los segmentos del texto que pertenecen al esquema argumentativo
de los que no. Una vez se tienen estos elementos, se clasifican en dependencia del esquema de anotación
usado. En esta etapa se obtienen los nodos y sus clasificaciones del grafo que conforman las estructuras 
argumentativas del texto. El siguiente paso constituye en la extracción de las relaciones entre los nodos.
Estas relaciones luego son clasificadas y dan información acerca de cómo interactuan las UDAs. Finalmente
el resultado es un grafo dirigido en donde se presenta un esquema argumentativo. La composición de este esquema
varía, aunque se evidencian ciertas similitudes entre estos, como la presencia de nodos afirmación y de evidencias
y también relaciones de apoyo y ataque.

\subsection{Propuestas}

En [\cite{palau2009argumentation}] se propone
el uso de modelos estadísticos como Naive Bayes y Support Vector Machines para la clasificación de 
oraciones en argumentativas o no y en su rol argumentativo en caso de que sea argumentativa, en este
caso se asume que las componentes argumentativas son oraciones completas. Para la predicción de relaciones
se usa un enfoque basados en reglas con la creación de una Gramática Libre de Contexto. Las representaciones
de las oraciones se ven dadas por atributos elegidos a mano dado el conocimiento experto sobre la argumentación
en el tema tratado, elementos como adverbios, verbos, signos de puntuación, palabras clave, estadísticas del texto
(Tamaño de oración, distancia media de palabras) son usados para la extracción y clasificación de las UDA, además
se usan también como base en la creación de las relgas para la extracción de relaciones.

[\cite{goudas2015argument}] al igual que [\cite{palau2009argumentation}] clasifica las oraciones como
argumentativas o no mediante el uso de diferentes clasificadores como Naive Bayes, Random Forest, Regresión
Logística y Support Vector Machines. En este trabajo se aumenta la grandularidad de la segmentación al permitir
la extracción de los segmentos que contienen la carga argumentativa de dentro de las oraciones previamente clasificadas
como tal, esto se realiza mediante la extracción de etiquetas BIO de las oraciones con el uso de un 
Conditional Random Field. La predicción de las relaciones como un problema de clasificación
usando Support Vector Machine para clasificar pares de UDA en relacionados o no. Atributos asignados a mano 
son usados en la extracción de UDA como posición de la oración en el texto, cantidad de verbos, comas, adverbios,
palabras, entidades en la oración, también se emplean gaceteras que guardan entidades relacionadas con el dominio 
especifico y palabras clave indicadoras de frases argumentativas. 

[\cite{stab2017parsing}] propone un mecanismo de segmentación basado en Conditional Random Fields. La clasificación
y predicción de relaciones es modelado conjuntamente como con dos clasificadores Support Vector Machine y un problema
de Optimización Lineal Entero que encuentra la mejor estructura y asegurar una disposición arborea. En la segmentación
de las UDA se extraen por cada token su posición en el texto, si precede o sucede a un signo de puntuación, su parte de
la oración, la probabilidad de que sea el comienzo de una UDA dado sus tokens anteriores, entre otros. Para la extracción
y clasificación de relaciones se proponen otros conjuntos de atributos como la cantidad de sustantivos comunes entre
las componentes fuente y el objetivo, la presencia de indicadores argumentativos, representaciones vectoriales de tokens.

En [\cite{eger2017neural}] se enfocaron en
modelar el problema como un problema de etiquetado de secuencias, usando Redes Neuronales Recurrentes como 
Long-Short Term Memory en versiones bidireccionales capturando información desde ambos lados de la secuencia,
extrayendo información morfológica de las palabras mediante Redes Neuronales Convolucionales teniendo una capa final de
Conditional Random Field. Realizaron experimentos al modelar el problema como uno de Dependency Parsing cuyo problema
consiste en construir un árbol de dependencia que codifique las estructuras argumentativas. El problema fue modelado
como un problema de reconocimiento de entidades nombradas, en donde las entidades son las UDA.

[\cite{galassi2021deep}] propone el uso de redes residuales y en combinación con mecanismos de atención
para la creación de un modelo el cual, conjuntamente, clasifica el tipo de UDA y la relación existentes entre estas.
Este trabajo define el concepto de distancia argumentativa, añadiéndolo como feature y asume que las UDA ya fueron 
extraídas. En este caso además de la distancia argumentativa las secuencias son representadas por sus representaciones
vectoriales de GloVe.

En [\cite{dykes2020reconstructing}] se propone métodos basados en reglas para la extracción de argumentos sobre
textos en Twitter, estos métodos se centran en la confección de reglas basadas en anotaciones lingüísticas como
partes de la oración, lemas de palabras. La recuperación está basada en los esquemas argumentativos comunes presentes
en los textos. Este método apunta a una mayor precisión prescindiendo de recobrado. 

Se contempan disímiles enfoques al problema de EA con desde una perspectiva enmarcada en modelos neuronales, estadísticos, secuenciales y end-to-end. 
En los modelos propuestos existen algunos problemas relacionados con
el uso independiente de modelos para el completamiento de la tareas asociadas a EA. Esto conlleva a la propagación
de errores ya que las entradas de etapas posteriores presentan las errores de las etapas anteriores.
También la no explotación de las interrelaciones entre variables debe ser tomado en cuenta, ya que se pierde
información computada en etapas anteriores. Modelos end-to-end y 
Multi-Task Learning han surgido para mitigar estos errores. En estos modelos el entrenamiento se hace de forma 
conjunta y no secuencial, previendo la propagación de errores y también se aprenden las distintas tareas al mismo
tiempo, pudiendo extraer las relaciones entre las diferentes tareas realizadas.
La arbitrariedad y dificultad de extraer de los atributos creados a mano persiste en varios modelos [\cite{goudas2015argument}, \cite{palau2009argumentation}]
además de que estos pueden ser difíciles de adaptar para otros documentos [\cite{eger2017neural}].
La creación de modelos que usen representaciones aprendidas sin o poca intervención de conocimiento humano hace 
que sean más robustos a los diferentes esquemas de argumentación pudiendo adaptarse a estos sin 
realizarle cambios en este.

\section{Proyección de corpus}

La EA no presentan corpus de grandes magnitudes y la gran mayoría de estos se encuentran en inglés o alemán. 
Se ha visto además que la creación de estos corpus es una tarea costosa. Dada la necesidad de trabajar en otros idiomas además de
inglés y alemán se han hecho esfuerzos de utilizar los datos existentes en tareas de idiomas en lenguajes 
diferentes al original. En este aspecto existen dos vertientes principales, proyección de corpus y transferencia
directa. En [\cite{eger2018cross}] se muestra que la proyección constituye una mejor alternativa sobre la transferencia
directa y presentan un algoritmo para realizar la proyección de las etiquetas dadas las alineaciones de palabras.
Con este algoritmo se transfieren las etiquetas de un corpus anotado a nivel de tokens en un lenguaje origen 
hacia su traducción en un lenguaje objetivo. El proceso se divide en varias partes.

\subsection{Traducción de oraciones}

La Traducción Automática consiste en el proceso de usar inteligencia artificial para
traducir texto de un lenguaje fuente a un lenguaje objetivo sin la intervención humana.
En la actualidad este campo ha dado un gran paso pasando de modelos estadísticos a modelos
neuronales llevando a tener traducciones de una alta calidad, condición necesaria para una
buena proyección. En [\cite{eger2018cross}] se
evidencia que los resultados obtenidos a partir de traducción automática y a partir de 
traducción humana no varían significativamente. 

Por lo anterior es posible la traducción de corpus hacia otros lenguajes mediante las
herramientas existentes obteniendo una buena representación final, la cual se puede usar en la creación de
oraciones alineadas, primer paso para la proyección de etiquetas.

\subsection{Alineación de palabras}

La alineación de palabras consiste en encontrar las palabras generadas en el lenguaje objetivo por las 
palabras en el lenguaje fuente.
Este problema es atacado en los modelos IBM, modelos estadísticos de traducción automática que usan estos 
índices para obtener más información. Estos modelos son principalmente Bayesianos y han sido la base de
otras herramientas como FastAlign [\cite{dyer2013fastalign}] que incorpora una reducción sustancial del
tiempo de cómputo y además de una mejora al modelo, EFMARAL [\cite{ostling2016efficient}] que usa
Cadenas de Markov Monte Carlo. Modelos más recientes se han enfocado en explotar las representaciones
vectoriales de palabras y el uso de métodos de atención para la extracción de las
alineaciones llegando convertirse en el estado del arte [\cite{dou2021word}].

\subsection{Proyección de etiquetas}

La proyección de etiquetas consiste en transportar las etiquetas de las palabras en la secuencia origen
hacia las palabras de la secuencia destino tomando como datos las alineaciones entre estas. En [\cite{yarowsky2001inducing}]
se trata el problema de proyección de frases nominales, estas frases tienen como característica que son resistentes
a ser divididas en caso de ser traducidas, dicha propiedad se cumple para las UDA también, que aunque evidencian 
cambios en el orden de las palabras, mantienen la misma ventana. La proyección de UDA son más simples en el caso
que solamente se tiene en cuenta la ventana y las etiquetas en estas son constantes, no pasa con la proyección en
frases nominales las cuales pueden cambiar dentro de una ventana, por lo que algoritmos más simples existen
para esta tarea [\cite{eger2018cross}].  


% % SECCION DE PROBLEMAS


% \textbf{CAMBIO DE SECCION} 
% % HABLAR SOBRE LOS PROBELEMAS POR SEPARADO: Segmentacion UDA, Clasificacion UDA, Relacion, Clasificacion Relacion

% La segmentación de las Unidades de Discurso Argumentativas es el primer problema se enfrenta.
% Para esto se han usado varios enfoques desde asumiendo que las oraciones en su totalidad argumentativas o 
% no [\cite{palau2009argumentation}], realizar procesos de dos etapas en las que primero se consideran oraciones 
% en argumentativas o no y luego se emplea un Conditional Random Field para delimitar las UDA 
% [\cite{goudas2015argument}]. El uso de Naive Bayes y Support Vector Machines en la clasificación de las UDA se 
% observó en [\cite{palau2009argumentation}, \cite{goudas2015argument}]. Este enfoque estadístico de segmentación y
% clasificación tiene como base una fuerte 

% Se ha propuesto modelos de Naive Bayes para la clasificación de UDA basados en atributos en la estructura
% y contenido de oraciones del texto además de realizar Gramáticas Libres de Contexto para extraer las 
% relaciones entre las UDA [\cite{palau2009argumentation}]. El uso de Conditional Random Fields para la 
% segmentación de las UDA es algo recurrente la literatura debido a su habilidad de modelar
% Support Vector Machine ha estado
% presente en propuestas que fueron consideradas estados del arte en su momento igual, basándose atributos

% Muchos de los esfuerzos realizados se vieron realizados en dominios específicos tales como textos legales [\cite{palau2009argumentation}],
% medicina [\cite{mayer2020transformer}], publicaciones científicas [\cite{lauscher2018inventor}, \cite{yang2018scidtb}], foros
% de debate [\cite{niculae2017argument}], noticias [\cite{sardianos2015news}] y ensayos argumentativos [\cite{stab2017parsing}]



% \textbf{CAMBIO DE SECCION} 

% \section{Argumentación}

% La argumentación es una actividad en donde el ponente trata de justificar racionalmente 
% sus puntos de vista sobre ciertos temas, con el objetivo de persuadir al receptor a aceptar
% las afirmaciones emitidas.  
% Con el tiempo se ha podido elaborar una teoría en la cual basarse para el estudio de esta. 

% \subsection{Argumentación Clásica}

% Las primeras citas de la argumentación provienen de la retórica encontradas en \emph{On Rhetoric} 
% escrito por Aristóteles, lo cual se puede considerar como la argumentación clásica. 
% En este el autor identifica tipos de persuación, a través de la credibilidad, lógica, 
% emoción, disposición del receptor o receptores al contenido.
% Además se introduce los componentes que presentan la argumentación clásica, tales como
% introducción, contexto y antecedentes, posición y argumentos del ponente, pruebas negativas o positivas
% que soporten los argumentos dados y conclusión y llamado a acción. [\cite{classicalArgument}] 

% \subsection{Argumentación Rogeriana}

% Este es visto en debates, en los cuales existen varios puntos de vista y se desea llegar a una 
% decisión mutualmente beneficiosa para todas las partes. Desarrollado por Carl Rogers, es basado en 
% la comprensión de los puntos de vista del oponenete teniendo en cuenta tres principios 
% [\cite{rogerianArgument}]:

% \begin{enumerate}
%     \item Escuchar y hacer entender al oponente que es entendido.
%     \item Encontrar validez en la posición del oponente.
%     \item Inducir al oponente que sus deseos no difieren de lo dicho por el ponente.
% \end{enumerate}

% \subsection{Método Toulmin}

% El Método Toulmin fue extrapolado del libro \emph{The Uses of Argument}[\cite{toulmin_2003}] escrito por Stephen E. Toulmin.
% Este método divide los argumentos en seis partes: afirmación (claim), fundamento (grounds), 
% justificación (warrant), calificador (qualifier), refutación (rebuttal) y respaldo (backing).
% Mediante las afirmaciones se conoce el argumento principal que el autor quiere probar a la audiencia,
% estas son respaldados con fundamentos siendo estos las evidencias y hechos en que se apoya el autor.
% Las justificaciones pueden estar explícitas o implícitas y son suposiciones que vinculan los
% fundamentos con las afirmaciones, estas a su vez pueden ser respaldadas por conocimiento.
% El esquema introduce la posibilidad de otra sitaución válida a la establecida en las afirmaciones
% mediante la refutación. Los calificadores son usados para dar más información de la calidad o seguridad
% de las afirmaciones dadas. Un ejemplo de este esquema es:

% \emph{[Se escucharon ladridos y aullidos en la distancia](fundamento), [probablemente](calificador) 
% [haya perros en las cercanías](afirmación).}

% En este ejemplo, además de las partes explícitas, se encuentran implícitas, la justificación 
% (los perros son animales que ladran y aullan), el respaldo (se sabe que existen perros en la zona) y 
% la refutación (Puede ser que hayan lobos o coyotes cerca). [\cite{toulminArgument}]

% Este método presenta una popular y definida forma de representar las estructuras de la argumentación, 
% además de haber sido estudiado y ampliado [\cite{freeman2011argument}] y usado en diferentes 
% investigaciones [\cite{stab2017parsing}, \cite{niculae2017argument}].


% \section{Extracción de Argumentos}

% La Extracción de Argumentos es una rama del Procesamiento del Lenguaje Natural encargado 
% de la extracción y anotación de estructuras argumentativas en textos de diferente índole.
% El modelo de Toulmin es una base teórica muy usada para la representación de dichas estructuras.
% De este se dice que un argumento como mínimo está compuesto por una afirmación sujeta a fundamentos.
% Se puede observar como un grafo en el cual los nodos son las UDA, ya sean afirmaciones o fundamentos, y
% las relaciones entre estos son los vértices. 
% Dada la definición anterior se puede observar los problemas fundamentales a resolver:

% \begin{itemize}
%     \item Extracción de las UDA .
%     \item Clasificación de las UDA.
%     \item Extracción de las relaciones entre las UDA
%     \item Clasificación de las relaciones extraídas.
% \end{itemize}

% Para las proposiciones existen varios tipos de clasificaciones, se han usado Afirmación Mayor, Afirmación y 
% Premisa [\cite{eger2017neural}, \cite{stab2017parsing}], otros no dependen de clases fijas y se ajustan 
% a la anotación del corpus con que se trabaja [\cite{galassi2021deep}]. El modelado de las relaciones
% tiene dos vertientes principales. Por una parte están las que restringen estas relaciones a tener una
% estructura arborea [\cite{eger2017neural}, \cite{stab2017parsing}], lo cual facilita su modelación y 
% comprensión, mientras que otras permiten una representación más libre y por lo tanto más compleja 
% [\cite{galassi2021deep}, \cite{niculae2017argument}].

% En las aproximaciones presentadas para atacar dicho problema se destacan los algoritmos divididos
% por etapas en la que cada subproblema es independientemente resuelto y su salida es la entrada del
% próximo subproblema [\cite{stab2017parsing}, \cite{goudas2015argument}], esta aproximación es problemática 
% debido a la propagación del error y en que no explotan las interrelaciones entre las variables [\cite{eger2017neural}]. 
% Otros tipos de algoritmos son los llamados end-to-end los cuales se enfrentan a los problemas presentados 
% anteriormente de manera conjunta, partiendo de los tokens de los textos haciendo el procesamiento de tal 
% forma que no se pierda información y que esta se pueda compartir entre etapas [\cite{eger2017neural}]. También 
% se encuentran modelos en los cuales se realiza Multi-Task Learning y son capaces de realizar varias tareas de 
% aprendizaje al mismo tiempo.

% Otro problema que se tiene es la representación de las componentes argumentativas. Existen versiones
% las cuales estas son extraídas por conocimiento experto del dominio a tratar, esto requiere de un procesamiento
% mayor y se corre el riesgo de no necesariamente obtener la mejor representación. Exite otro enfoque el cual consiste
% en aprender la representación automáticamente y haciendo usos de representaciones previamene calculadas de palabras.

% El estado del arte actualmente en este campo se ha logrado gracias al uso de técnicas de aprendizaje de
% máquina. En este ámbito se han creado modelos basados en Support Vector Machine, Recurrent Neural Networks, 
% Residual Neural Networks, Attention Neural Networks, Long-Short Term Memory. Se ha modelado el problema de 
% extracción de argumentos como problemas de inferencia en un factor graph [\cite{niculae2017argument}], 
% parseo de dependencia, etiquetado de secuencia, extracción de entidades nombradas y sus relaciones [\cite{eger2017neural}].

% El modelo más promisorio se basa en redes neuronales con módulo de atención y redes residuales. Lo interesante
% es que el uso de features dependientes del problema es nula, sus features dependen solo de los argumentos en sí
% usando embeddings de BERT. Los resultados de este modelo superan el estado del arte que existía en su actualidad
% y deja un camino abierto al perfeccionamiento de este mediante la inclusión de features dependientes de contextos
% específicos.

% \subsection{Métricas}

% Las métricas usadas generalmente para evaluar los modelos consisten en precisión, recobrado y F1.
% Estas métricas son usadas en las tareas de segmentación de UDA, clasificación de UDA, predición de enlaces, clasificación de enlaces,
% ya que todas se pueden ver como problemas de clasificación.  
% Aunque existen otras métricas que son posibles de calcular que proveen más información. 
% Para esto se definie las métricas 50\%F1 y 100\%F1 las cuales son métricas a nivel de segmentos, estas
% significan que si el segmento coincide en un porciento mayor o igual al señalado se considera que se pudo identificar
% correctamente el segmento. La razón detrás de estas métricas es que las UDA a veces tienen comienzo o finales ambiguos
% por lo que se da un margen para este error.

% \section{Corpus}

% Los corpus para EA se caracterizan por ser de pequeño formato y de ser heterogeneos en sus estilos de anotación.
% Entre los encontrados se encuentran:

% \begin{itemize}
%     \item SciDTB [\cite{yang2018scidtb}]
%     \item Persuasive Essays [\cite{stab2017parsing}]
%     \item AbstRCT [\cite{mayer2020transformer}]
%     \item Dr. Inventor [\cite{lauscher2018inventor}]
%     \item Cornell eRulemaking Corpus [\cite{niculae2017argument}]
% \end{itemize}

% \section{Proyección de corpus}

% La EA es un campo que no posee grandes cantidades de corpus anotados y los encontrados 
% no se encuentran en el idioma español.
% La traducción ha demostrado ser costosa incluso para pequeños corpus [\cite{eger2018cross}],
% lo cual hace que haga falta una herramienta que automatice la obtención de estos a partir
% de un lenguaje inicial, mayormente inglés. 
% La proyección de etiquetas es una técnica en la cual se transfieren las etiquetas de 
% un corpus anotado a nivel de tokens en un lenguaje origen hacia su traducción en un lenguaje
% objetivo.
% El proceso se divide en varias partes.

% \subsection{Traducción de oraciones}

% La Traducción Automática consiste en el proceso de usar inteligencia artificial para
% traducir texto de un lenguaje fuente a un lenguaje objetivo sin la intervención humana.
% En la actualidad este campo ha dado un gran paso pasando de modelos estadísticos a modelos
% neuronales llevando a tener traducciones de una alta calidad. En [\cite{eger2018cross}] se
% evidencia que los resultados obtenidos a partir de traducción automática y a partir de 
% traducción humana no varían significativamente. 

% Para el proceso de proyección de etiquetas se establecen una serie de suposiciones que son
% cumplidas en los corpus trabajados. Una de estas es que las UDA no cruzan límites oracionales, esto
% permite hacer una separación de los textos en oraciones sin perder información de los segmentos. 

% Por lo anterior es posible la traducción de corpus hacia otros lenguajes mediante las
% herramientas existentes obteniendo una buena representación final, la cual se puede usar en la creación de
% oraciones alineadas, primer paso para la proyección de etiquetas.

% \subsection{Alineación de palabras}

% La alineación de palabras consiste en encontrar las relaciones entre las palabras de una secuencia
% en un lenguaje origen con las palabras de la traducción de dicha secuencia en un lenguaje objetivo.
% Este problema es atacado en los modelos IBM, modelos estadísticos de traducción automática que usan estos 
% índices para obtener más información. Estos modelos son principalmente Bayesianos y han sido la base de
% otras herramientas como FastAlign [\cite{dyer2013fastalign}] que incorpora una reducción sustancial del
% tiempo de cómputo y además de una mejora al modelo, EFMARAL [\cite{ostling2016efficient}] que usa
% Cadenas de Markov Monte Carlo. Modelos más recientes se han enfocado en explotar las representaciones
% de palabras vectoriales ( Word Embeddings?) y el uso de modelos de atención para la extracción de las
% alineaciones llegando convertirse en el estado del arte \cite{dou2021word}.

% \subsection{Proyección de etiquetas}

% La proyección de etiquetas consiste en transportar las etiquetas de las palabras en la secuencia origen
% hacia las palabras de la secuencia destino tomando como datos las alineaciones entre estas. En [\cite{yarowsky2001inducing}]
% se trata el problema de proyección de frases nominales, estas frases tienen como característica que son resistentes
% a ser divididas en caso de ser traducidas, dicha propiedad se cumple para las UDA también, que aunque evidencian 
% cambios en el orden de las palabras, mantienen la misma ventana. La proyección de UDA son más simples en el caso
% que solamente se tiene en cuenta la ventana y las etiquetas en estas son constantes, no pasa con la proyección en
% frases nominales las cuales pueden cambiar dentro de una ventana, por lo que algoritmos más simples existen
% para esta tarea [\cite{eger2018cross}].  

