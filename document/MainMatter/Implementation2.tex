\chapter{Experimentación}\label{chapter:implementation}

% Explicación de las métricas: F1, %F1, Precision, Recobrado
% Con que software se hizo el modelo: Keras, Tensorflow
% Explicacion de corpus estudiados: Decir que los esquemas de argumentacion presentes son diferentes, por lo tanto
% usaron varios para ver cual era el mejor que se acomodaba al conjunto de validación final (Granma)
% Experimentación con segmenter
% Experimentación con link prediction
% Conclusiones: Mejor clasificar y segmentar UDA con segmenter y dejar la clasificacion de UDA como tarea auxiliar en 
% link prediction, poner los mejores ejemplos y peores ejemplos para los mejores modelos.

En el cap'itulo se describe los conjuntos de datos utilizados para el entrenamiento y validación del modelo 
propuesto. Se mencionan las herramientas utilizadas para la confección del software. Adem'as se describe el 
proceso de experimentación y evaluación de los modelos constru'idos con los conjuntos de datos. Finalmente
se presentan los resultados obtenidos con los modelos al aplicarles las cartas a la dirección extra'idas 
del periódico Granma. 

\section{Conjuntos de Datos}

Para el entrenamiento de los modelos propuestos se utilizaron corpus diferentes, estos
presentas esquemas de anotación distintos entre sí. Dado que todos los corpus estaban en inglés, 
se les aplicó previamente el algoritmo de proyección de corpus propuesto y se crearon sus respectivas
versiones en español. A los conjuntos de entrenamientos se le aplicó la técnica de \emph{backtranslation}
para aumentar su tamaño. Como conjunto para validar los modelos fueron usadas las cartas a la dirección
extraídas del periódico Granma.

\subsection{Ensayos Argumentativos}\label{corpus:persuasive_essays}

Este corpus [TODO \cite{CORPUS PAPER}] presenta unos 402 documentos, dividido por los autores en 286 documentos para entrenamiento (70\%), 
80 para prueba (20\%) y 36 para validación (10\%). Los contenidos de estos son ensayos de estudiantes sobre temas
como por ejemplo: cooperar o competir, contribuciones de la tecnología a la sociedad.
Las anotaciones de las UDA se conforman por segmentos 
de textos considerados argumentativos, estos segmentos son clasificados en MajorClaim (751), Claim (1506) y Premise (3832).
La estructura de las relaciones entre las UDA conforman árboles en los que se tienen como raíz las 
MajorClaim del texto. Las relaciones solo estan permitidas entre Premise-Premise y Premise-Claim, clasificadas
en de ataque (219) o apoyo (3613). Las relaciones entre Claim y MajorClaim son anotadas de manera diferente, por medio de 
darle a las Claim una clasificación de si está a favor (1228) o en contra (278) de las MajorClaim del documento.
Para el análisis de este corpus se consideraron las relaciones Claim-MajorClaim de igual manera que las otras.

TODO Tabla con las metricas de este modelo dado los diferentes modelos propuestos

\subsection{CDCP}\label{corpus:cdcp}

TODO PONERLE NUMEROS A LOS PORCIENTOS

El corpus [TODO \cite{CORPUS PAPER}] está conformado por 731 comentarios de usuarios extraídos de la web bajo el tema de 
prácticas de cobro de deudas a los consumidores (CDCP por sus siglas en ingles Consumer Debt Collection Practices).
Las UDA se encuentran segmentadas en oraciones y todas se consideran argumentativas, estas son clasificadas en 
POLICY (17\%), VALUE (45\%), FACT (16\%), TESTIMONY (21\%) y REFERENCE (1\%). Las relaciones se encuentran 
clasificadas en REASON (97\%) y EVIDENCE (3\%), aunque los datos presenta una baja conexión entre ellos, con 
aproximadamente el 3\% de los posibles pares anotados.

\subsection{AbsTRCT}

TODO 

\subsection{Aumento de Datos}

\emph{Backtranslation} se aplicó a los textos originales en inglés, como lenguaje intermedio se utilizó el español.
Una vez se tenía la traducción se utilizó el algoritmo de proyección de etiquetas para el anotado de las etiquetas sobre
las versiones nuevas.
Al aplicar \emph{backtranslation} sobre los conjuntos de datos su tamaño se duplicó aunque los textos se mantuvieron
similares. 

TODO Métricas de comparación entre los textos traducidos y los originales.

\subsection{Cartas a la dirección}

Las cartas a la dirección constituyen un segmento del periódico Granma en los cuales son publicadas
cartas enviadas por la población o empresas a dicha entidad. En general, las cartas 
presentan dudas o problemas de la población con el objetivo de obtener respuestas del organismo
asociado. Se extrajeron 2891 cartas desde el 30 de agosto del 2013 hasta el 28 de octubre del 2022. Se 
presentan aproximadamente 975000 palabras en los datos, en promedio, la cantidad de palabras por carta es de 330.
Se encontraron 874 cartas en respuesta a cartas enviadas lo que representa un 30\% del total. Se extrajeron
los comentarios asociados a las cartas, en este sentido 987 cartas no presentan comentarios y en promedio 
se realizan 2 comentarios por carta. Los textos presentan un título y un formato relativamente libre, 
aunque en las cartas de respuesta se puede observar una firma de la persona que respondió y la entidad que 
representa. Del total de cartas se selecciaron las que fueran en respuesta a otra y también las 
cartas que fueron respondidas para tener una mayor concentraci'on de cartas que fueran argumentativas, 
esta selección est'a conformada por 1702 cartas lo que representa un 59\% del total de cartas.

TODO SI HAY TIEMPO ANOTAR LAS CARTAS CON ALGUN ESQUEMA SENCILLO (Claim, Premise) (Attack, Support) y Mapear 
las clasificaciones de los otros modelos a estas para tener algun tipo de evaluacion.

\section{Implementación}

La implementación de los modelos y algoritmos de procesamiento y visualización de datos se encuentran en 
un repositorio de GitHub\footnote{\url{https://github.com/luisoibarra/argument-mining}}. Esta implementación
está concebida para que se pueda extender fácilmente para el uso con otros idiomas además del inglés y el 
español. Se basa en una arquitectura de procesamiento secuencial en el cual cada paso del proceso realiza
una tarea específica y lo más desacoplada posible de las otras. Las tareas realizadas son:

\begin{itemize}
    \item Creación del corpus en un formato estándar. Dado que los corpus vienen en diferentes formas, este paso se realiza para trabajar sobre una misma representación de este.
    \item Proyección del corpus de un lenguaje fuente a un lenguaje objetivo.
    \begin{itemize}
        \item Traducción y alineación de oraciones.
        \item Alineación de palabras.
        \item Proyección de etiquetas.
    \end{itemize}
    \item Extracción y clasificación de UDAs.
    \item Extracción y clasificación de las relaciones entre las UDAs.
    \item Visualización de los resultados.
\end{itemize}

\subsection{Herramientas}

El lenguaje empleado para la confección del software fue \textbf{Python} [\cite{TODO }], este presenta 
una gran variedad de herramientas 
para el trabajo con texto, visualización de datos y la creación de modelos de aprendizaje profundo.
Se utilizó \textbf{tensorflow} [\cite{TODO}] en su versión 2.9.2 para la construcción y entrenamiento de los modelos. 
Para el procesamiento de los textos se utilizaron \textbf{nltk} [\cite{TODO}] y \textbf{spacy} [\cite{TODO}], con estos se realizaron tareas
como la extracción de tokens y oraciones del texto, la anotación de las etiquetas POS. Se utilizaron 
ambas bibliotecas para el procesamiento debido que en dependencia de la sitaución cada una presenta diferentes
ventajas. En el caso de \textbf{nltk} esta presenta algoritmos rápidos para el procesamiento de texto que no 
requieren de muchos recursos computacionales, sin embargo, estos algoritmos no están disponibles de inmediato
para otros lenguajes como el español. \textbf{Spacy} por su parte presenta algoritmos más certeros a costo 
de mayor tiempo de procesamiento y gasto de recursos computacionales y también presenta una cantidad de lenguajes 
disponibles mucho mayor. Para la visualización, manejo de los datos y cálculo de métricas se utilizaron 
\textbf{matplotlib} [\cite{TODO}], \textbf{pandas} [\cite{TODO}] y \textbf{sklearn} [\cite{TODO}]. 
Para la recolección de las cartas del periódico Granma se utilizó \textbf{scrapy} [\cite{TODO}].

\subsection{Visualización de resultados}

Como interfaz visual para el usuario se utilizó la herramienta Brat [TODO \cite{}]. Esta herramienta permite
la visualización y edición de las estructuras argumentativas. Dado que Brat es una página web, esta se puede
desplegar y permitir su uso online.  

\section{Experimentación}

Para realizar la selección del modelo se utilizó el corpus de Ensayos Argumentativos. Con este se ajustaron
las arquitecturas e hiperparámetros de los modelos propuestos. La mejor combinación de estos fue utilizada para el entrenamiento de
los corpus restantes. Finalmente los modelos fueron utilizados para anotar las cartas a la dirección. 

\subsection{Hardware}

Gran parte del procesamiento se llevo a cabo en una computadora $i5$ con $8GB$ de RAM ampliada con $4GB$ de memoria 
swap [TODO \cite{Memoria swap}], aunque se requirió el uso de la plataforma \textbf{Colab} [TODO \cite COLAB] para 
el entrenamiento de algunos modelos por falta de recuros locales.

\subsection{Segmentador de UDA}

Para el segmentador se probaron variantes de la arquitectura asignando valores de hiperparámetros a los propuestos
por las investigaciones originales. (TODO Poner las asignaciones de hiperparametros que no varían). Las variaciones 
de la arquitectura se vieron dadas por la presencia o no de lo siguientes elementos:

\begin{itemize}
    \item Atributos de POS en la entrada del algoritmo.
    \item Atributos extraídos por la CNN de la palabra.
    \item Atributos features extraídos por la LSTM bidireccional de la palabra.
    \item Conexiones residuales.
    \item Capa densa final.
    \item Capas de normalizaciones.
\end{itemize}

Los resultados obtenidos son:

TODO FOTOS o TABLAS con los resultados relevantes en Persuassive Essays

Se eligió el modelo con (TODO poner el modelo escogido) debido a los resultados presentados.

TODO FOTOS o TABLAS con los resultados relevantes en los OTROS CORPUS

TODO Mostrar ejemplos buenos y malos
TODO Decir opiniones de los errores
TODO Hacer analisis curvas de aprendizaje -> overfitting, underfitting, representación de los datos en los conjuntos de validacion/entrenamiento

\subsection{Predictor de Enlaces}

Se entrenaron diferentes variantes de arquitecturas con diferentes hiperparámetros. Los hiperparámetros
iniciales fueron los propuestos por la investigación original y también fueron probados algunas variantes 
de estos. La arquitectura varía por la presencia del módulo de atención en esta.

Se construyeron varios conjuntos de hiperparámetros. Los resultados fueron:

TODO PONER TABLA con los conjuntos de los hiperparámetros

TODO FOTOS o TABLAS con los resultados relevantes en Persuassive Essays 

El modelo elegido fue (TODO poner el modelo escogido).

TODO FOTOS o TABLAS con los resultados relevantes en los OTROS CORPUS

TODO Mostrar ejemplos buenos y malos
TODO Decir opiniones de los errores
TODO Hacer analisis curvas de aprendizaje -> overfitting, underfitting, representación de los datos en los conjuntos de validacion/entrenamiento

\subsection{Modelo conjunto}

Dado que la clasificación de UDAs es hecha tanto en el segmentador como en el predictor de enlaces es necesaria 
la selección de cómo se va a desambiguar esta clasificación. En base a los resultados obtenidos en el proceso
de experimentación se prefirió el resultado devuelto por el segmentador de UDAs.

TODO Poner estadisticas de ambas aproximaciones

\section{Validación}

Se entrenaron varios modelos en los diferentes corpus con los hiperparámetros seleccionados. Con estos modelos 
se procesaron las cartas y se observó cual era el que más se ajustaba a las estructuras argumentativas presentes 
en estas. Dado que la definición de UDA es algo que varía, es complejo realizar un método que evalúe de forma 
justa los resultados obtenidos por los diferentes modelos de manera conjunta. TODO PREGUNTAR COMO HACER LA VALIDACION?

\subsection{Ensayos Argumentativos}

TODO Resultados obtenidos en el Granma con el modelo.
TODO Mostrar ejemplos buenos y malos

\subsection{CDCP}

TODO Resultados obtenidos en el Granma con el modelo.
TODO Mostrar ejemplos buenos y malos

\subsection{AbsTRCT}

TODO Resultados obtenidos en el Granma con el modelo.
TODO Mostrar ejemplos buenos y malos

\subsection{Consideraciones}

TODO Poner cual modelo se considera mejor para el Granma 
TODO Decir opiniones de los errores. (El texto de Granma posee una estructura diferente al entrenante y mas libre)
