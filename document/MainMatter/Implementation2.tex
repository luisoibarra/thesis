\chapter{Detalles de Implementación y Experimentos}\label{chapter:implementation}

% Explicación de las métricas: F1, %F1, Precision, Recobrado
% Con que software se hizo el modelo: Keras, Tensorflow
% Explicacion de corpus estudiados: Decir que los esquemas de argumentacion presentes son diferentes, por lo tanto
% usaron varios para ver cual era el mejor que se acomodaba al conjunto de validación final (Granma)
% Experimentación con segmenter
% Experimentación con link prediction
% Conclusiones: Mejor clasificar y segmentar UDA con segmenter y dejar la clasificacion de UDA como tarea auxiliar en 
% link prediction, poner los mejores ejemplos y peores ejemplos para los mejores modelos.

\section{Conjuntos de Datos}

Para el entrenamiento de los modelos propuestos se utilizaron corpus diferentes, estos
presentas esquemas de anotaci'on distintos entre s'i. Como conjunto de validación fueron
usados cartas extra'idas del peri'odico Granma. Dado que todos los corpus estaban en ingl'es, 
se les aplic'o previamente el algoritmo de proyecci'on de corpus propuesto y se crearon sus respectivas
versiones en espannol.

\subsection{Ensayos Argumentativos}\label{corpus:persuasive_essays}

Este corpus presenta unos 402 documentos, dividido por los autores en 286 documentos para entrenamiento (70\%), 
80 para prueba (20\%) y 36 para validación (10\%). Los contenidos de estos son ensayos de estudiantes sobre temas
como por ejemplo: cooperar o competir, contribuciones de la tecnologia a la sociedad.
Las anotaciones de las UDA se conforman por segmentos 
de textos considerados argumentativos, estos segmentos son clasificados en MajorClaim (751), Claim (1506) y Premise (3832).
La estructura de las relaciones entre las UDA conforman 'arboles en los que se tienen como ra'iz las 
MajorClaim del texto. Las relaciones solo estan permitidas entre Premise-Premise y Premise-Claim, clasificadas
en de ataque (219) o apoyo (3613). Las relaciones entre Claim y MajorClaim son anotadas de manera diferente, por medio de 
darle a las Claim una clasificaci'on de si est'a a favor (1228) o en contra (278) de las MajorClaim del documento.
Para el an'alisis de este corpus se consideraron las relaciones Claim-MajorClaim de igual manera que las otras.

\subsection{CDCP}\label{corpus:cdcp}

TODO PONERLE NUMEROS A LOS PORCIENTOS

El corpus est'a conformado por 731 comentarios de usuarios extra'idos de la web bajo el tema de 
pr'acticas de cobro de deudas a los consumidores (CDCP por sus siglas en ingles Consumer Debt Collection Practices).
Las UDA se encuentran segmentadas en oraciones y todas se consideran argumentativas, estas son clasificadas en 
POLICY (17\%), VALUE (45\%), FACT (16\%), TESTIMONY (21\%) y REFERENCE (1\%). Las relaciones se encuentran 
clasificadas en REASON (97\%) y EVIDENCE (3\%), aunque los datos presenta una baja conexión entre ellos, con 
aproximadamente el 3\% de los posibles pares anotados.

\subsection{AbsTRCT}

\subsection{DrInventor}

\subsection{Cartas a la direcci'on}

Las cartas a la direcci'on constituyen un segmento del peri'odico Granma en los cuales son publicadas
cartas enviadas por la poblaci'on o empresas a dicha entidad. En general, las cartas 
presentan dudas o problemas de la poblaci'on con el objetivo de obtener respuestas del organismo
asociado. Se extrajeron las cartas CANTIDAD DE CARTAS desde FECHA INICIAL hasta FECHA FINAL. Se encontraron
CANTIDAD DE CARTAS RESPONDIDAS lo que representa un PORCIENTO DE CARTAS RESPONDIDAS del total. 

TODO SI HAY TIEMPO ANOTAR LAS CARTAS CON ALGUN ESQUEMA SENCILLO (Claim, Premise) (Attack, Support) y Mapear 
las clasificaciones de los otros modelos a estas para tener algun tipo de evaluacion.

\section{Experimentación}

Para realizar la selecci'on del modelo se utiliz'o el corpus de Ensayos Argumentativos. Con este se ajustaron
las arquitecturas e hiperpar'ametros de los modelos propuestos. El mejor modelo luego fue entrenado con los
corpus restantes. Todos los modelos finales fueron utilizados para anotar las cartas a la direcci'on. (TODO ver 
qu'e criterio considero para ver cual anotaci'on fue la mejor en las cartas a la direcci'on)

\subsection{Software}

Para realizar los modelos se construy'o una imagen de Docker con las herramientas necesarias. Se utiliz'o 
Tensorflow 2.9.2 para la construcci'on y entrenamiento de los modelos. (TODO Seguir agregando cosas) 

\subsection{Segmentador de UDA}

Para el segmentador se probaron variantes de la arquitectura asignando valores de hiperparametros a los propuestos
por las investigaciones originales. (TODO Poner las asignaciones de hiperparametros que no varian). Las variaciones 
de la arquitectura se vieron dadas por la presencia o no de lo siguientes elementos:

\begin{itemize}
    \item Atributos de POS en la entrada del algoritmo
    \item Atributos extra'idos por la CNN de la palabra.
    \item Atributos features extra'idos por la LSTM bidireccional de la palabra.
    \item Conexiones residuales
    \item Capa densa final
    \item Capas de normalizaciones
\end{itemize}

Los resultados obtenidos son:

TODO FOTOS o TABLAS con los resultados relevantes en Persuassive Essays

Se eligi'o el modelo con (TODO poner el modelo escogido) debido a los resultados presentados.

TODO FOTOS o TABLAS con los resultados relevantes en los OTROS CORPUS

TODO Mostrar ejemplos buenos y malos
TODO Decir opiniones de los errores

\subsection{Predictor de Enlaces}

Se entrenaron diferentes variantes de arquitecturas con diferentes hiperpar'ametros. Los hiperpar'ametros
iniciales fueron los propuestos por la investigaci'on original y tambi'en fueron probados algunas variantes 
de estos. La arquitectura var'ia por la presencia del m'odulo de atención en esta.

Se construyeron varios conjuntos de hiperpar'ametros. Los resultados fueron:

TODO PONER TABLA con los conjuntos de los hiperpar'ametros

TODO FOTOS o TABLAS con los resultados relevantes en Persuassive Essays 

El modelo elegido fue (TODO poner el modelo escogido).

TODO FOTOS o TABLAS con los resultados relevantes en los OTROS CORPUS

TODO Mostrar ejemplos buenos y malos
TODO Decir opiniones de los errores

\subsection{Modelo conjunto}

Dado que la clasificación de UDAs es hecha tanto en el segmentador como en el predictor de enlaces es necesaria 
la selecci'on de c'omo se va a desambiguar esta clasificaci'on. En base a los resultados obtenidos en el proceso
de experimentaci'on se prefiri'o el resultado devuelto por el segmentador de UDAs.

\section{Validaci'on}

\subsection{Ensayos Argumentativos}

TODO Resultados obtenidos en el Granma con el modelo.
TODO Mostrar ejemplos buenos y malos

\subsection{CDCP}

TODO Resultados obtenidos en el Granma con el modelo.
TODO Mostrar ejemplos buenos y malos

\subsection{Consideraciones}

TODO Poner cual modelo se considera mejor para el Granma 
TODO Decir opiniones de los errores. (El texto de Granma posee una estructura diferente al entrenante)
