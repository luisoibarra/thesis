\chapter{Propuesta}\label{chapter:proposal}

El modelo propuesto se divide en dos secciones. En la primera sección se realiza la segmentación y clasificación de 
las UDAs como tareas conjuntas. En la segunda sección se predicen los enlaces y sus clasificaciones, tomando
como tareas auxiliares la clasificación de las UDAs. Dada la heterogeneidad de las conjuntos de datos disponibles 
en EA, los modelos poseen un mínimo de atributos, esto permite que el modelo por sí solo aprenda 
la mejor representación para el esquema de anotación con que se entrene.

\section{Segmentación y clasificación de UDAs}

Esta primera parte se modela como un problema secuencia a secuencia cuyo objetivo es asignar a los tokens 
extraídos del documento entrada una etiqueta BIOES para segmentar las UDAs. Para la clasificación del tipo 
de UDA, al conjunto de etiquetas BIES se le añadieron las clasificaciones que presenta el corpus entrenante.
En el siguiente ejemplo se muestra una salida del modelo presentando las clasificaciones de
$A$ como argumento y $P$ como premisa:

\begin{adjustwidth}{25pt}{25pt}
    En$_O$ primer$_O$ lugar$_O$ ,$_O$ [\emph{el$_{B-A}$ correo$_{I-A}$ electrónico$_{I-A}$ puede$_{I-A}$ 
    contar$_{I-A}$ como$_{I-A}$ uno$_{I-A}$ de$_{I-A}$ los$_{I-A}$ resultados$_{I-A}$
    más$_{I-A}$ beneficiosos$_{I-A}$ de$_{I-A}$ la$_{I-A}$ tecnología$_{I-A}$ moderna$_{E-A}$}] .$_{O}$ 
    [\emph{Años$_{B-P}$ atrás$_{I-P}$ ,$_{I-P}$ las$_{I-P}$ personas$_{I-P}$ pagaban$_{I-P}$ gran$_{I-P}$ cantidad$_{I-P}$ 
    de$_{I-P}$ dinero$_{I-P}$ para$_{I-P}$ enviar$_{I-P}$ sus$_{I-P}$ cartas$_{I-P}$ y$_{I-P}$ sus$_{I-P}$ 
    pagos$_{I-P}$ estaban$_{I-P}$ sujetos$_{I-P}$ al$_{I-P}$ peso$_{I-P}$ de$_{I-P}$ sus$_{I-P}$ 
    cartas$_{I-P}$ o$_{I-P}$ paquetes$_{I-P}$ y$_{I-P}$ muchos$_{I-P}$ accidentes$_{I-P}$ podrían$_{I-P}$
    causar$_{I-P}$ problemas$_{I-P}$ que$_{I-P}$ causarían$_{I-P}$ que$_{I-P}$ el$_{I-P}$ correo$_{I-P}$ 
    no$_{I-P}$ fuera$_{I-P}$ enviado$_{E-P}$}] .$_{O}$
\end{adjustwidth}

\subsection{Modelo de segmentación y clasificación de UDAs}

Sea $D$ un documento entrada, este es separado en una secuencia de $n$ tokens $D_i$ donde $n$ es la mayor longitud encontrada
en los documentos del conjunto de datos (si la cantidad de tokens es menor que $n$ entonces $D_i$ es completado con un token especial de enmascarado). 
A cada token se le asigna
su representación vectorial GloVe de dimensión $g=300$, dando como resultado $G_{ij} \in \mathbb{R}^{n \times g}$.
Esta representación inicial presenta información semántica de las palabras y conserva las relaciones 
espaciales entre ellas. 

Para la representación de información morfológica de la palabra se construyen dos
codificadores que procesan los caracteres de cada token y devuelven una representación vectorial de estos.
A cada caracter se le asigna un vector que será entrenado convirtiendo un token en un vector de dimensión
$q \times c$, donde $q$ es el tamaño máximo de palabra en el conjunto de datos y $c$ es la dimensión del vector
asignado a cada caracter.
Uno de estos modelos está basado en CNN, este modelo entrena una representación de caracteres de dimensión
$cd=50$ representando un token como un vector de dimensión $q \times cd$. Se conforma por una capa de convolución unidimensional
con $f=30$ filtros y un kernel de tamaño $k=3$, seguida por una capa \emph{max pooling} que convierte la secuencia en un vector
de dimensión $1 \times f$, que luego es concatenado a la representación del token a que pertenece.
Otro modelo utilizado para calcular una representación morfológica se encuentra basado en RNN. Se usó
un modelo LSTM bidireccional con dimensión $l=25$ para calcular la representación del token, para las dimensiones de los caracteres se
utilizan vectores de tamaño $l$, el resultado final constituye la concatenación de la corrida hacia adelante y
hacia atrás formando una representación de dimensión $1 \times 2 \cdot l$ del token. Este vector es concatenado a la representación
del token correspondiente. Otro atributo usado en la representación de los tokens constituyen las etiquetas de 
Partes de la Oración de estos.
El conjunto de etiquetas elegido es un conjunto universal [\cite{petrov2011universal}] aplicable a cualquier idioma.
De estas etiquetas se les extrae la codificación \emph{one-hot} y esta es transformada por una capa densa con $p=5$ neuronas
y función de activación \emph{ReLU}, el resultado es concatenado a la representación del token correspondiente. Mediante 
la extracción de estos atributos el token es representado en tres maneras, semántica, morfológica y estructural, con el 
objetivo de que sean aprendidas los rasgos lingüísticos correspondientes a la tarea.

Del proceso de vectorización sale un vector con dimensión $n \times t$ donde $t$ es la dimensión final de la representación
de los tokens. Este vector es modificado por una capa LSTM bidireccional de dimensión $m=200$, a esta salida se le 
añade una conexión residual al ajustarle la dimensión con una capa densa. Luego, la secuencia es procesada por una 
capa densa de dimensión $k=100$ con activación \emph{ReLU} produciendo una representación final de dimensión 
$n \times k$. Finalmente, se utiliza una capa CRF
para la clasificación final de la secuencia en las etiquetas finales. El resultado final constituye un vector
de dimensión $n$ que representa las clasificaciones inferidas por el modelo (Figura \ref{fig:segmenter_model}).

Para prevenir el sobreajuste se agregaron capas de normalización y de \emph{dropout} entre cada proceso y se usaron regularizaciones
L2 y \emph{dropout} en las capas densas y LSTM, el valor asignado al dropout es de $0.5$. 
Para prevenir el sobreentrenamiento se aplicó una 
terminación temprana de este cuando no se encontraba una mejora de la función de pérdida en el conjunto de validación
por más de 10 épocas consecutivas. Como optimizador se utilizó Adam con una tasa de aprendizaje de $0.001$.

\subsection{Posprocesamiento de segmentación y clasificación de UDAs}

La salida del modelo constituye una secuencia de etiquetas en formato BIOES. Esta está propensa
a contener errores en su formato, por ejemplo, secuencias no terminadas en E, segmentos continuos con más de una 
meta-etiqueta, entre otros.
Para la corrección de la estructura se propone el siguiente algoritmo con dos partes. La primera
consiste en arreglar la estructura BIOES, para esto se mantiene una ventana de tamaño
3, [\_ , \_ , \_], sobre la secuencia y se asume que la parte anterior a la posición de la ventana no presenta errores. Al encontrar una
ventana inválida se necesita observar la siguiente ventana para poder decidir cómo se arregla el error, ya que se
podría dar el caso que se observe [O, O, I] y la próxima sea [O, I, O], en donde solamente viendo la primera ventana no se podría saber si el cambio 
correcto corresponde a sustituir I por B o por S. Una vez observadas las dos ventanas, se procede a realizar el 
arreglo correspondiente. En casos donde sea ambigua la manera de arreglar la ventana, [I, I, O] por ejemplo
(La I o la O pueden ser 
sustituidas por una E), se utiliza una función que recibe un segmento y devuelve la gravedad del error.
El error con mayor gravedad será arreglado, en caso de ser iguales se arreglará la etiqueta más a la izquierda.
Este procedimiento devuelve una secuencia BIOES correctamente anotada, debido a que a partir de una secuencia sin 
errores en cada paso se va arreglando la ventana y una vez esta llega al final arregló todos los elementos de la secuencia.
Una vez la secuencia tiene la estructura BIOES correctamente anotada el problema
consiste en arreglar las meta-etiquetas, ya que una misma secuencia BIOES pudo haber sido anotada con diferentes
tipos, en este caso se toma la etiqueta más representativa del segmento continuo.

\begin{figure}[p]
	\begin{center}
		\begin{center}
            \includesvg[scale=.65]{Graphics/Modelo_Segmenter_UDA.svg}
        \end{center}
	    \caption{Segmentador UDAs.}\label{fig:segmenter_model}
	\end{center}
\end{figure}

\newpage

\section{Predicción y clasificación de enlaces}

En esta segunda parte el modelo se encarga de las tareas de extracción y clasificación de enlaces.
El problema consiste en clasificar pares de UDAs, representando origen y objetivo del enlace, 
en el tipo de relación que existen entre estas.
Como tarea auxiliar se clasifican los tipos de UDAs que intervienen en la relación. La salida 
del modelo constituye en una tupla de tres elementos: la clasificación de la relación, 
la clasificación de la UDA origen, la clasificación de la UDA objetivo. Si el enlace existe o no 
es calculado partir de la clasificación de la relación.   

\subsection{Modelo de predicción y clasificación de enlaces}

Sean dos UDAs, $S$ y $T$, donde $S$ representa la fuente de la relación, mientras que $T$ representa
al objetivo. Estas secuencias son tokenizadas y se les asigna la representación GloVe de cada palabra, obteniendo
dos vectores de dimensión $u \times g$, donde $u$ es el tamaño máximo de UDAs en el conjunto de entrenamiento
y $g=300$ es la dimensión del \emph{embedding}.
Estos vectores son modificados por una red densa compuesta por $ca = 4$ capas con activación \emph{ReLu}
de dimensiones $50$, $50$, $50$, $300$, añadiendo una conexión residual a la salida de esta. 
El próximo paso consiste en aplicar una capa densa de dimensión $di=50$ y luego un \emph{average pooling}
de tamaño $dp=10$, obteniendo vectores de dimensión $\frac{q}{dp} \times di$. 
Estos vectores son modificados por un LSTM bidireccional con $lm=50$ unidades. Un módulo de atención es aplicado 
sobre los vectores fuentes, 
en este actúan como consultas el promedio de los vectores objetivo y como llaves y valores los vectores fuentes,
el procedimiento simétrico es realizado para los vectores objetivos.
La salida de los procesamientos son concatenados con la distancia argumentativa obteniendo una representación 
conjunta de la relación a analizar. Esta representación conjunta es modificada por una red residual obteniendo
una representación final de dimensión $l=20$ y luego sometida a los clasificadores de relación y de tipos de UDAs
(Figuras \ref{fig:link_predictor_model1} y \ref{fig:link_predictor_model2}).

Para prevenir el sobreajuste se agregaron capas de normalización y de \emph{dropout} entre cada 
proceso y se usaron regularizaciones L2 y \emph{dropout} en las capas densas y LSTM, 
todos los \emph{dropout} tienen valor $dr=0,1$. Para prevenir el sobreentrenamiento se aplicó una 
terminación temprana de este cuando no se encontraba una mejora de la función de pérdida en el 
conjunto de validación durante $v=5$ épocas consecutivas. Como optimizador se utilizó Adam con descenso 
exponencial con tasa de aprendizaje $lr=0.003$.

Dado que se realiza un aprendizaje de varias tareas, se tienen varias funciones de pérdida individuales que conforman 
la función de pérdida final $e$. Sea $e_r$ la función de pérdida de la clasificación de la relación, $e_s$ la del tipo de UDA origen  
y $e_t$ del tipo de UDA objetivo, entonces $e = 10 \cdot e_r + e_s + e_t$.

\subsection{Preprocesamiento de predicción y clasificación de enlaces}

El uso de este modelo se concreta a nivel de documento, en donde ya se tienen las UDAs extraídas. Para alimentar
al modelo con los pares de UDAs y sus distancias argumentativas se seleccionan todos los pares de estos que cumplan
que no se enlacen con ellos mismos, por ejemplo, $a \rightarrow a$; y que su distancia argumentativa absoluta sea menor 
que $da=10$. Estas restricciones disminuyen el número de pares extraídos por documentos a una cantidad lineal 
con respecto a la cantidad de UDAs, presentes ya que por cada UDA solamente se tomarían $2 \cdot da$ elementos como máximo
(los que la preceden y los que la suceden). 

Además de las etiquetas $R$ originales del conjunto de entrenamiento, se añaden elementos extras a este
conjunto. Estos elementos son las representaciones inversas de las relaciones, por ejemplo, si $a \xrightarrow{c} b$ entonces 
se agregará el par $b \xrightarrow{c^{-1}} a$, donde $c^{-1}$ es una nueva clasificación de relación que representa
el inverso de la clasificación $c$. Este proceso se realiza para aumentar la cantidad de relaciones positivas en el
conjunto entrenante, ya que aun con las reducciones hechas existe un desbalance de clases positivas y negativas en
las relaciones.

\subsection{Posprocesamiento de predicción y clasificación de enlaces}

Se calcula una salida extra a partir de las distribuciones de probabilidad de las relaciones 
devueltas por el modelo, esta salida representa si el par está enlazado o no directamente, para este cálculo se 
suman las categorías vinculadas a las clases originales del conjunto de datos, esta se toma como la probabilidad de estar 
enlazados, que en caso de ser mayor del 50\% se devuelve verdadero. Para dar el resultado final se eliminan del 
conjunto de respuesta las relaciones anotadas con las etiquetas inversas añadidas en el paso de preprocesamiento 
y son devueltas aquellas que se clasifiquen como enlazadas según el criterio anterior.

\newpage

\begin{figure}[p]
    \begin{center}
        \includesvg[scale=.6]{Graphics/Modelo_Link_Prediction1.svg}
        \caption{Predictor de enlaces.}\label{fig:link_predictor_model1}
    \end{center}
\end{figure}
\begin{figure}[p]
    \begin{center}
        \includesvg[scale=.6]{Graphics/Modelo_Link_Prediction2.svg}
    \end{center}
    \caption{Predictor de enlaces (continuación).}\label{fig:link_predictor_model2}
\end{figure}
