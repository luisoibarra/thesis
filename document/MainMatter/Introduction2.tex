\chapter*{Introducción}\label{chapter:introduction}
\addcontentsline{toc}{chapter}{Introducción}

% \cite{brownlee2018better}\\
% \textcite{brownlee2018better}\\
% \parencite{brownlee2018better}\\
% \citet{brownlee2018better}
% \citep{brownlee2018better}

% Contexto histórico-social donde se desarrolla
% Cuales son las condiciones historicas que hicieron necesarias la creacion de la tesis, o que dieron origen a la problematica de esta

La argumentación es una actividad verbal, social y racional destinada a convencer 
a un crítico razonable de la aceptabilidad de un punto de vista mediante la presentación 
de una constelación de proposiciones que justifican o refutan la proposición expresada 
en el punto de vista [\cite{van2004systematic}]. Esta definición concentra en ella 
partes escenciales de lo que es la argumentación, dicha actividad se encuentra presente
en varias facetas de la cotidianedad humana, en la escritura o lectura de documentos,
en las interacciones sociales de las personas, está presente cada vez que se
plantea un argumento y se trata de que este triunfe en un debate donde 
se exponen elementos que lo apoyen.   

En la actualidad es necesario tener acceso a la información necesaria
de forma rápida y simple. Esto no siempre es posible dado la gran cantidad de información existente y
que es generada en cada momento. En caso de tener una vía de acceder a esta se podrían realizar acciones
con mayor rapidez y calidad. Con la argumentación se podría hacer explícitas las razones de las personas 
al afirmar algo sobre un tema teniendo así su punto de vista individual, y con suficientes personas, colectivo.

% "La argumentación es una actividad verbal, social y racional destinada a convencer 
% a un crítico razonable de la aceptabilidad de un punto de vista mediante la presentación 
% de una constelación de proposiciones que justifican o refutan la proposición expresada 
% en el punto de vista." [\cite{van2004systematic}]. Esta definición concentra en ella 
% partes escenciales de lo que es la argumentación, dicha actividad se encuentra presente
% en varias facetas de la cotidaniedad humana, en la escritura o lectura de documentos,
% en las interacciones sociales de las personas, se encuentra presente cada vez que se
% plantea un argumento y se trata de que este triunfe por medio de un debate en la que 
% se muestran los elementos que lo apoyen.  


% Antecedentes del problema, justificación y motivación. 
% Antecedentes: 
%   Tareas similares que se quedan cortas: Opinion Mining, Citation Mining, Controversy Detection
%   Cómo se ha estudiado primero a mano. Trabajo dificil.
% Justificacion:
%   Limitaciones del trabajo manual
%   Se quiere hacer el trabajo automatico.
%   El proyecto esta integrado en un proyecto nacional  reconocido CORESPUC

Varias tareas en el Procesamiento de Lenguaje Natural (PLN) se han desarrollado en 
torno a diferentes problemas relacionados con 
la argumentación. Entre tales tareas se encuentra el minado de opiniones, el cual es el 
estudio computacional de las opiniones, sentimientos y emociones expresadas en un texto 
[\cite{liu2010sentiment}], la detección de controverisas enfocada en la identificación de 
temas y textos en donde puntos de vista conflictivos están presentes; y también la zonifiación
argumentativa quel clasifica las oraciones por su rol argumentativo dentro de un artículo
científico. Estas tareas solamente muestran cuáles opiniones, puntos de conflictos y roles 
de argumentos 
presenta el texto, pero lo realizan de manera separada y no muestran el porqué de estas. 
Para lograr extraer esta información faltante es necesario realizar un 
análisis de los argumentos dados, para pasar de texto no estructurado a datos argumentativos 
que den entendimiento de los puntos de vista y de cómo se apoyan o atacan entre sí. Este análisis
es posible realizarlo mediante métodos convencionales manuales o utilizando programas
especializados para la anotación, aunque la práctica ha demostrado que este proceso requiere 
de una gran cantidad de tiempo y de personal calificado. Con la inmensa cantidad de datos 
que se genera a diario este análisis es impracticable de realizar de forma manual, por esto se 
estudian y crean métodos encargados de automatizar esta tarea.

% Breve presentación de la problemática. 
%   Por todo lo anterior nace la EA, rama de PLN ..., definir tareas de la EA, tocar los enfoques realizados
% (No es el estado del arte aunque se puede hablar un poco de él) Elementos involucrados en el punto de vista cientifico, lleva corpus.

La Extracción de Argumentos (EA) nace como la rama del PLN encargada
del estudio de métodos para la extracción automática de las estructuras argumentativas de 
los textos y su posterior procesamiento [\cite{lawrence2020argument}]. Esta tarea se divide en 
cuatro subtareas fundamentales: i) la extracción y ii) clasificación de las componentes 
argumentativas del texto, y iii) la extracción y 
iv) clasificación de las relaciones entre estas. Estas tareas han sido abordadas de diferentes maneras,
desde modelos secuenciales [\cite{palau2009argumentation}, \cite{goudas2015argument}] hasta 
\emph{end-to-end} [\cite{eger2017neural}], desde el uso de clasificadores clásicos 
como Naive Bayes o SVM [\cite{niculae2017argument}, \cite{stab2017parsing}] hasta el uso de 
aprendizaje profundo [\cite{galassi2021deep}, \cite{mayer2020transformer}], y sigue 
siendo un área creciente de estudio.

% Actualidad, novedad e importancia teórica y práctica. 
%   En español no tiene mucho estudio. Citar casos del español, para Cuba ninguno
%   Creación de un corpus anotado el cual se podrá mejorar con el tiempo

La EA se caracteriza por la falta de conjuntos de datos y 
por la heterogeniedad que presentan estos a la hora de decidir cómo realizar las 
anotaciones, además que la gran mayoría de los estudios realizados en el campo se encuentran en 
un reducido conjunto de idiomas como el inglés, alemán o chino [\cite{eger2018cross}]. 
En el español, pocas intervenciones se han dado en el análisis de los argumentos [\cite{esteve2020mineria}] y en 
lo correspondiente a Cuba no se encontró ninguna referencia. 

% Diseño teórico.
%    Problema: 
%    Objeto de Investigación: Procesamiento de Lenguaje Natural
%    Campo de acción: Linguistica computacional
%    Hipótesis o preguntas científicas
%    Objetivos generales y específicos
%       General: Diseño e implementación de un algoritmo para el estudio de la argumentación en el periódico digital Granma
%       Especifico Construcción del corpus de los periódicos: Crawler, Anotación (scpaCy).
%       Especifico Implementacion de la interfaz gráfica para consultar los resultados.

Por ello esta tesis tiene como objetivo proponer un algoritmo para 
la extracción y análisis de estructuras argumentativas en textos 
de la prensa cubana, constituyendo el primero de su tipo en lo que respecta a la búsqueda del autor. 
Para lograr dicho objetivo se necesitan realizar varias operaciones.
En primer lugar, es necesario obtener los textos a analizar del sitio 
web del periódico Granma. Luego, se necesita confeccionar algoritmos capaces de realizar las tareas 
de EA sobre estos textos, para estos algoritmos es necesaria la confección de conjuntos 
de datos en español en los cuales se puedan entrenar. Finalmente, se requiere de una interfaz visual 
en la que sea posible la interacción de los usuarios, en tareas como visualización y edición, 
con los resultados obtenidos. 

Se propone la construcción de un programa capaz de realizar la extracción de los textos del periódico Granma, 
en específico, se enfoca en su sección de Cartas a la Dirección. Para la extracción
de argumentos se presentan dos modelos, el primero se encarga de la segmentación y clasificación
de las componentes argumentativas. Este es un modelo secuencia a secuencia en la cual las palabras 
son vectorizadas por sus características morfológicas mediante el uso de redes neuronales convolucionales
(CNN por sus siglas en inglés \emph{Convolutional Neural Network}) y \emph{Long Short Term Memory} (LSTM), por su información
semántica mediante \emph{embeddings} \emph{Global Vectors} (GloVe) y por su información 
estructural mediante su parte de la oración, estas 
secuencias son procesadas por una red LSTM bidireccional y sus atributos son usados por una capa 
de \emph{Conditional Random Field} (CRF) 
para su clasificación final en etiquetas B (inicio de segmento), I (dentro del segmento), O (afuera de segmento), 
E (fin de segmento) y S (segmento de un elemento) (BIOES) con información adicional para la clasificación de las 
unidades de discurso argumentativas (UDA). El segundo modelo está encargado de la extracción y clasificación 
de las relaciones entre las UDAs. En él se utilizan las representaciones GloVe de las palabras en las secuencias 
y su distancia argumentativa para la clasificación de una tupla en la que su primer elemento 
constituye el texto candidato de donde parte la relación o fuente y el segundo, el texto candidato a recibir la 
relación u objetivo, en el tipo de relación existente entre estos elementos. La entrada es procesada mediante 
una LSTM bidireccional y módulos de atención cruzada entre los elementos de la fuente y los elementos del objetivo.
Para la visualización se crea un ambiente de desarrollo con la herramienta Brat [\cite{brat}], la cual permite realizar las 
tareas requeridas a los datos anotados por los modelos.

En resumen, se presenta un estudio de la argumentación\ español al realizar la extracción de 
argumentos en prensa, además se crea un conjunto de datos anotado que puede servir
para un estudio más profundo sobre los esquemas argumentativos presentes en estos textos. 

%    Estructura del trabajo

El presente documento está dividido en cuatro capítulos.
En el Capítulo \ref{chapter:argumentation} se presentan las definiciones necesarias para una introducción
al tema de la argumentación y la EA.
El Capítulo \ref{chapter:background} presenta un recorrido por los 
conceptos, métodos y arquitecturas relacionadas con el Aprendizaje Automático para el problema 
en cuestión, además de realizar un compendio y análisis de los enfoques utilizados para dar solución
a los problemas asociados a la EA y el procedimiento de proyección de corpus, utilizado para la 
traducción de conjuntos de datos del inglés al español. 
Luego, en Capítulo \ref{chapter:proposal}
se describe la propuesta de los modelos para realizar la EA.
En el Capítulo \ref{chapter:implementation} se mencionan los diferentes conjuntos de datos utilizados 
en los experimentos y los resultados obtenidos en las tareas realizadas.
Finlamente, se exponen conclusiones y recomendaciones sobre el trabajo 
realizado.
