\chapter*{Introducción}\label{chapter:introduction}
\addcontentsline{toc}{chapter}{Introducción}

La teoría de la argumentación es el estudio interdisciplinario de cómo las conclusiones
pueden ser apoyadas o socavadas por premisas a través de razonamiento lógico [\cite{wiki-arg-theory}].
Esta es usada en varios aspectos de la vida como en las negociaciones, debates públicos, 
publicaciones científicas, enseñanza, leyes. Esta data sus inicios en la \dots

La Extracción de Argumentos (EA) es la rama de Procesamiento del Lenguaje Natural encargada
del estudio de métodos para la extracción automática de las estructuras argumentativas de 
los textos y su posterior procesamiento. Con el aumento creaciente de fuentes de información
se hace cada día más necesaria la existencia de métodos que sean capaces de identificar estas
estructuras sin la intervención directa de humanos. Esta rama en la práctica encuentra su uso
en el reconocimiento de contenido engañoso (TODO https://doi.org/10.1162/coli_a_00338), apoyo de 
decisiones médicas [\cite{mayer2020transformer}] y legales, y reforzamiento de debates de
diferentes temas [\cite{niculae2017argument}]. El proceso de extracción está compuesto por 
la extracción y clasificación de los componentes argumentativos y de las relaciones entre estos. 
