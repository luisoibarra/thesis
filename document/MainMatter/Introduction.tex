\chapter*{Introducción}\label{chapter:introduction}
\addcontentsline{toc}{chapter}{Introducción}

La teoría de la argumentación o argumentación es el estudio interdisciplinario de cómo las conclusiones
pueden ser apoyadas o socavadas por premisas a través de razonamiento lógico (CITE Wikipedia English Argumentation Theory).
Esta es usada en varios aspectos de la vida como en las negociaciones, debates públicos, publicaciones
científicas, enseñanza, leyes. Todo argumento se compone escencialmente en un conjunto de premisas,
un método de razonamiento y una conclusión.


La Extracción de Argumentos es la rama del Procesamiento de Lenguaje Natural encargada de formularizar
y modelar dicho problema para su posterior procesamiento. Los modelos existentes generalmente se ecargan de
extraer y clasificar las componentes argumentativas y sus relaciones de una fuente no estructurada de 
texto, capturando un proceso de razonamiento y argumentación en su estructura final. Los métodos
de realizar este procedimiento varían en dependencia de las características en que se trabaja y el objetivo
a que se quiere llegar. Entre los métodos usados se pueden mencionar métodos ad-hoc basados en gramáticas,
que aprovechan atributos del texto como etiquetas POS y etiquetas de NER, que codifican diferentes 
patrones argumentativos (CITE Reconstructing Arguments from Noisy Text). Dicho enfoque requiere de trabajo 
humano para la creación de una gramática que permita resultados satisfactorios. Este tipo de enfoque
generalmente sacrifica recobrado por precisión y no permite una generalización del problema, contribuyendo
a no ser muy escalable. Otro método es el basado en técnicas de aprendizaje de máquina. Estos métodos están
divididos en diferentes vertientes de acuerdo a cómo procesan los datos. Uno de ellos lo realiza por
etapas, en el cual en cada una se resuelve independientemente los subproblema y la salida de la etapa 
anterior es entrada a la etapa siguiente. Las etapas en que generalmente se divide el problema
son, separación de unidades argumentativas y no argumentativas, clasificación de las unidades 
argumentativas y la identificación de estructuras argumentativas (CITE Identifying Argumentative Discourse Structures in Persuasive Essays, cualquier otro que use este enfoque).
Este enfoque trae consigo una modularidad elevada al resolver las tareas de manera independiente, pero
tiene la desventaja que los errores de etapas anteriores son pasados a las siguientes y además los algoritmos
usados no ven todo el contexto del texto pudiendo perder características que permitirían un mejor resultado.
Otro enfoque usado son los llamados end-to-end, en estos el modelo entrenado aprende los pasos para convertir
directamente la entrada del algoritmo en la salida deseada al entrenar sus diferentes partes de manera 
simultanea. En el contexto de EA la entrada serían los tokens de los textos y la salida sería las
estructuras argumentativas anotadas en dependencia del algoritmo usado. Este enfoque mitiga las posibles
deficiencias del enfoque por etapas, al juntar todo el proceso en una sola eliminando la propagación
del error, además de que al tener todos los datos es posible encontrar mayor cantidad de correlaciones 
entre ellos, también no requiere de una ingeniería de atributos (features) tan elaborada (CITE Neural End-to-End Learning for Computational Argumentation Mining).

En la práctica la EA tiene un gran número de aplicaciones: (CITE Argument Mining Linguistic Foundation pag 5)
\begin{itemize}
    \item Análisis de opinión: Ayuda no solo a saber si la opinión es favorable o no, sino a saber
    porqué es favorable o no.
    \item Análisis de debate: Ayuda a detectar estrategias argumentativas
    \item Detección de incoherencias en un conjunto de argumentos y justificaciones
\end{itemize}


En EA la gran mayoría de las investigaciones y corpus se basan en el idioma inglés o alemán. Hay
una cantidad pequeña de investigaciones en español (CITE Minería de argumentación en el Referéndum del 1 de Octubre de 2017) 
y basados específicamente en prensa no se encontró ninguna referencia. En Cuba no se tiene constancia
tampoco de investigaciones realizadas sobre el tema por lo que se considera la primera investigación
sobre las estructuras argumentativas en el país. El trabajo forma parte de una colaboración entre la 
Facultad de Artes y Letras de la Universidad de La Habana y la Facultad de Matemática y Computación 
de la misma universidad mediante el proyecto CORESPUC (TODO Poner el nombre completo del proyecto?).
Mediante este trabajo se podrá (TODO Poner acá para qué se quiere hacer el trabajo)


Este trabajo necesita encontrar \emph{qué modelos se pueden usar en el español para la extracción de 
argumentos en textos, especialmente en la prensa} (TODO Posible Pregunta Científica?). Recientemente se ha introducido modelos basados 
en Transformer y Attention (CITE Transformer-Based Argument Mining for Health Care Application, El doctorado)
que han llegado a alcanzar resultados iguales o superiores al estado del arte de su momento (TODO Posible Hipótesis?). 
Por otro lado dado que los corpus están en inglés se desea saber si \emph{es factible usar métodos
para poder usar el conocimiento aprendido de los algoritmos entrenados en inglés en el español} 
(TODO Posible Pregunta Científica).


El objetivo principal de este trabajo se basa sobre \emph{el diseño e implementación de un algoritmo para 
el estudio de la argumentación en el periódico digital Granma} (Objetivo). Para esto primero
es necesaria la construcción de un corpus sobre el periódico. Para esto se realizará una recolección
de noticias mediante técnicas de scrapping para conformar un corpus inicial no anotado. Este corpus
será anotado por un modelo previamente entrenado usando técnicas de proyección entre lenguajes (CITE Cross-lingual)
y otros tipos de anotaciones tales como partes de la oración y entidades presentes.

Actualmente existe un corpus parcialmente
anotado y con algunos errores documentados en la plataforma CQPweb (TODO Alguna referencia al corpus?,
Correcto poner CQPweb?) así que en una primera parte se realizará correcciones pertinentes a dicho 
corpus y luego se anotarán otros atributos relevantes como la clasificación de las secciones del 
periódico en de opinión o no. Luego se aumentará dicho corpus mediante técnicas de scrapping 
(TODO en español?) y se verificará la aparición o no de noticias entre la versión PDF y la versión 
online del periódico. Se necesita además la implementación de una interfaz visual en la cual se 
pueda consultar (TODO Consultar qué?) el corpus.


El trabajo está conformado por \dots (TODO Poner la estructura del trabajo)


% Esqueleto
% \begin{itemize}
%     \item Hablar sobre el conocimiento y el pensamiento del ser humano como ser racional.
%     No existe una verdad única, si no diferente tipos de verdades para diferentes grupos de personas.
%     Cada grupo de personas presentan argumentos por los cuales creen esas verdades y no creen otras.
%     \item Introducir el tema de la argumentación en el NLP, su usos actuales e importancia.
%     \item Introducción de la problemática (Formar un corpus de Granma con estructuras argumentativas),
%     el porqué se quiere hacer esto (Justificación del proyecto CORESPUC, crear un estudio en español del tema)
%     \item Hablar sobre la importancia teórica y práctica del trabajo. No existen estudios en español,
%     los corpus en español son escasos. Ayuda a resolver la justificación de CORESPUC
%     \item Planteo de los objetivos y las preguntas científicas (Crear corpus argumentativo en Español y un framework para la extracción de argumentos en periódicos)
%     \item Estructura del trabajo
% \end{itemize}


% Aspectos que debe tratar la introducción (Se deben de decir implícitamente en los párrafos):

% \begin{itemize}

%     \item Contexto histórico-social donde se desarrolla
%     \item Antecedentes del problema, justificación y motivación. Cómo se ha estudiado primero a mano y luego computacionalmente el problema en la prensa. Motivacion, el proyecto esta integrado en un proyecto nacional  reconocido CORESPUC, lo cual tiene una justificación también.
%     \item Breve presentación de la problemática. (No es el estado del arte aunque se puede hablar un poco de él) Elementos involucrados en el punto de vista cientifico, lleva corpus.
%     \item Actualidad, novedad e importancia teórica y práctica. Revisar literatura (Actualidad, en español no tiene mucho estudio), Se propone un modelo computacional para estudiar ese asutnto que se han propuesto poco, para Cuba no hay ninguno y poco de investigación. 
%     \item Diseño teórico.

%     \begin{itemize}

%         \item Problema: 
%         \item Objeto de Investigación: Procesamiento de Lenguaje Natural
%         \item Campo de acción: Linguistica computacional
%         \item Hipótesis o preguntas científicas
%         \item Objetivos generales y específicos

%         \begin{itemize}
%             \item General
%             \begin{itemize}
%                 \item Diseño e implementación de un algoritmo para el estudio de la argumentación en el periódico digital Granma
%             \end{itemize}
%             \item Específicos
%             \begin{itemize}
%                 \item Construcción del corpus de los periódicos: Crawler, Anotación (scpaCy).
%                 \item Arreglar etiquetas en corpus activo con archivos VRT.
%                 \item Clasificar las secciones en de opinión o no.
%                 \item Verificar la aparición o no de noticias entre la versión PDF y la versión online del periódico.
%                 \item Implementacion de la interfaz gráfica para consultar los resultados.
%                 \item Lograr interoperailidad de la plataforma CQPweb.
%             \end{itemize}
%         \end{itemize}

%     \end{itemize}

%     \item Estructura del trabajo

% \end{itemize}