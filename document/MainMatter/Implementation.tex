\chapter{Detalles de Implementación y Experimentos}\label{chapter:implementation}

% \begin{itemize}

%     \item No interesa el cómo se hizo
%     \item Mostrar resultados
%     \begin{itemize}

%         \item Puede ser con estadísticas
%         \item Por casos de uso

%     \end{itemize}
%     \item Sección de metacódigo en caso de ser relevante

% \end{itemize}

% Experimentos y Discusión: diseño de los experimentos, dataset, como preprocesó, cómo se hizo la 
% construcción del corpus. Lo más detallado posible para que alguien pueda replicarlo. 
% Compración entre variantes de solución propias y las del estado del arte que sean comparables.

\section{Corpus}

\subsection{Procesamiento de Corpus}

Los corpus pueden venir en diferentes formatos, por esto se construyeron parsers con el objetivo
de llevar a una representación común todos los corpus. Esta representación está constituida por
un archivo anotado en formato CoNLL cuyas anotaciones son:

\begin{itemize}
    \item [B|I]-[Tipo de UDA](-[Tipo de relación]-[Distancia Argumentativa entre Objetivo y Origen])*
    \item O
\end{itemize}

En la conformación de este formato común se realizan una serie de procesamientos, estos son la 
tokenización del texto y la separación de oraciones. Ambos procesamientos pueden ser hechos 
mediante SpaCy y NLTK. 

\subsection{Aumento de Datos}

Dado que la cantidad de datos es relativamente pequeña es recomendable hacer uso de técnicas de 
aumento de datos. Uno de los métodos usados es la traducción automática de las oraciones para 
un lenguaje intermedio y su traducción de vuelta al lenguaje inicial. Los resultados obtenidos
muestran que los nuevos datos presentan una alta similitud entre sus palabras.

TODO Hacer métricas de comparación de strings entre la versión original y la versión aumentada.
Distancia de Jacard, de Levenshtein (versión de palabras)


\cite{feng2021data}

\subsection{Argumentative Essays}

Este corpus posee un esquema de anotación en el cual las UDA se clasifican como MajorClaim, Claim y 
Premise. Las relaciones se encuentran entre Premise-Claim y Premise-Premise las cuales son clasificadas
como support o attack. Las Claims presentan una posición o stance acerca de las MajorClaim en el 
texto correspondiente, estas son clasificadas en For y Against. Dado su similitud de las stance con
las relaciones y la necesidad de incluir las MajorClaim en estas se agregó las relaciones Claim-MajorClaim
haciendo uso de sus posiciones, la posición For se agregó como support y la posición Against como attack.

\section{Proyección de corpus}

\subsection{Traducción automática}

En la traducción automática se utilizaron los servicios de Google Translate mediante el paquete
deep-translate [TODO referencia al repo \cite{}]. Dicho paquete provee la capacidad de usar 
otras herramientas para la traducción, como (TODO Poner otras herramientas para la traducción)
aunque la mayoría de estas presentan algún tipo de restricción, como la necesidad de una subscripción
o un token para el uso de sus servicios. El uso de Google Translate no presenta estas restricciones
y puede ser usado sin ninguna configuración previa.

El procesamiento de esta etapa es guardado en formato de texto conteniendo las 
secuencias separadas por un separador en cada linea en donde la primera secuencia se
encuentra en el lenguaje inicial y al segunda se encuentra en el lenguaje final.

\subsection{Alineación de palabras}

Para la alineación de palabras se usan FastAlign y AwesomeAlign. El primero constituye
una vía rápida y de pocos recursos para la tarea, mientras que el segundo produce mejores
resultados con un coste de memoria y tiempo mayor. (TODO Poner las métricas de la página de
github de AwesomeAlign)

La salida de este proceso constituye un archivo en donde cada linea representa la 
alineación de palabras correspondiente al par de oraciones en el lenguaje fuente y objetivo. 

\subsection{Proyección de etiquetas}

La proyección de etiquetas constituye el último paso en la proyección de corpus. Para esto
se usan las alineaciones de palabras y oraciones para el paso de las etiquetas. Para esto
se usa el algoritmo propuesto en [\cite{eger2018cross}] el cual consiste en (TODO buscar en
que consiste el algoritmo)

Esta etapa crea un nuevo corpus con las etiquetas proyectadas en el lenguaje objetivo. El 
corpus se presenta en formato CoNLL.

\section{Segmentación de UDAs}

La entrada en el la segmentación es texto del cual se desean extraer las UDAs.

\subsection{Extracción de Features}

La cadena de texto es tokenizada y es representada como una secuencia 
de los vectores GloVe de cada uno de los tokens, dicha representación inicial 
luego es modificada en dependencia de los hiperparámetros del modeo. 
El tamaño de la secuencia es elegido como el tamaño de la secuencia máxima en el 
conjunto de entrenamiento.

Además de la representación de las palabras mediante GloVe también se calculan otros features
encargados de extraer información de la morfología de las palabras al analizar sus caracteres.
Para esto se aprende una representación vectorial de los caracteres mediante el uso de Redes Convolucionales
y Redes Neuronales Recurrentes. En concreto se utilizó max pooling para la red convolucional
y una LSTM bidireccional para la recurrente, en ambos casos el resultado fue concatenado
a la representación inicial GloVe.

Otro feature que se le añade es las etiquetas POS, dado que las etiquetas POS pueden
cambiar en dependencia del lenguaje y el modelo está pensado para ser utilizado entre
varios lenguajes se elige un conjunto de etiquetas universal para la representación.
(TODO Poner las etiquetas o un link hacia donde se encuentran). A la representación
de cada palabra es concatenado una representación densa de codificación one-hot de las 
etiquetas POS, dicha representación posee dimensión 5.

\subsection{Experimentos}



\subsection{Resultados}

Se procesa la salida para que sea una secuencia BIOES válida

\section{Predicción de enlaces}

\subsection{Extracción de Features}

GloVe Embeddings

\subsection{Resultados}
