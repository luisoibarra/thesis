\chapter{Argumentación}\label{chapter:argumentation}

En el captítulo se aborda la argumentación, definiciones y marcos teóricos existentes para el estudio de esta.
Luego se introduce la Extracción de Argumentos como campo de la lingüísticas computacional encargado del estudio 
y procesamiento de las estructuras argumentativas de textos. También se definen y explican los componentes y las tareas 
asociadas al problema de realizar la extracción de argumentos.

\section{Argumentación}

La argumentación es un tema tratado desde la antigüedad, Aristóteles lo defendía como la 
habilidad de, dada una pregunta, considerar los elementos útiles para persuadir a alguien, algo
similar a la retórica. De una perspectiva más contemporánea surgen las ideas de [\cite{perelman1969rhetoric}] 
enfocadas en un análisis de la retórica en donde se estipula que la teoría de la argumentación
responde a provocar o aumentar la adhesión de las personas a las tesis presentadas, por medio de 
técnicas discursivas. En [\cite{toulmin_2003}] se considera como argumento todo aquello que ofrece, 
o todo lo que es utilizado, para justificar o refutar una proposición. En este último se toma 
una perspecitva más racional y deductiva de la argumentación, dando como resultado lo que se 
conoce como el Método de Toulmin. 

\subsection{Método de Toulmin}

Este método divide los argumentos en seis partes: afirmación 
(\emph{claim}), fundamento (\emph{grounds}), justificación (\emph{warrant}), calificador 
(\emph{qualifier}), refutación (\emph{rebuttal}) y respaldo (\emph{backing}).
Mediante las afirmaciones se conoce el argumento principal que el autor quiere probar a la audiencia,
estas son respaldadas con fundamentos siendo estos las evidencias y hechos en que se apoya el autor.
Las justificaciones pueden estar explícitas o implícitas y son suposiciones que vinculan los
fundamentos con las afirmaciones, estas a su vez pueden ser respaldadas por conocimiento.
El esquema introduce la posibilidad de otra sitaución válida a la establecida en las afirmaciones
mediante la refutación. Los calificadores son usados para dar más información de la calidad o seguridad
de las afirmaciones dadas. Un ejemplo\footnote{Extraído de \cite{toulminArgument}} de este esquema es:

\begin{adjustwidth}{25pt}{25pt}
    [\emph{Se escucharon ladridos y aullidos en la distancia}]$_{fundamento}$, 
    [\emph{probablemente}]$_{calificador}$ 
    [\emph{haya perros en las cercanías}]$_{afirmación}$.
\end{adjustwidth}

En este ejemplo, además de las partes explícitas, se encuentran implícitas como la justificación 
(los perros son animales que ladran y aullan), el respaldo (se sabe que existen perros en la zona) y 
la refutación (Puede ser que hayan lobos o coyotes cerca).

Este método crea una definición compacta que ayuda a los investigadores a enfocar su búsqueda basados
en las diferentes categorías definidas. Además engloba de manera comprensible un tema tan complejo 
como la argumentación al tomar en cuenta gran parte de los elementos presentes en el razonamiento
realizado para llegar a conclusiones, incorporando incluso elementos probabilísticos en el proceso. 

\subsection{Rasgos lingüísticos}

Los rasgos lingüísticos son aquellas características que se encuentran presentes en los textos 
que hacen que estos obtengan la clasificación de argumentativos [\cite{venegas2005hacia}]. Con 
la identificación de estos se hace la tarea de extracción más sencilla y con un marco teórico 
que respalde las decisiones tomadas. Ejemplos de rasgos presentes en textos argumentativos:

\begin{enumerate}
    \item Marcas de orden que introducen párrafos: Primero, segundo, por un lado, por otra parte,
    finalmente.
    \item Comillas y citas: Citar palabras que refuercen la intervención recurriendo a autoridades
    o personajes.
    \item Nexos que expresan causa o consecuencias: ya que, porque, pues, con motivo de, gracias a,
    considerando que, por lo tanto, de manera que.
\end{enumerate}

Estos rasgos además de dar indicación de la existencia de argumentos dan pie para conocer las relaciones
entre estos y los tipos de argumentos. Por ejemplo, \emph{por lo tanto}, implica que lo que viene 
a continuación es una conclusión la cual es apoyada por lo dicho anterior en el texto, algo parecido
sucede con \emph{ya que}, en este caso implica que lo siguiente es un argumento que se encuentra 
relacionado con lo mencionado antes.

En [\cite{venegas2005hacia}] se determinan 16 categorías y 51 rasgos lingüísticos, dando una idea 
de la gran variedad de marcadores presentes en la argumentación.

\section{Extracción de Argumentos}

El PLN es un subcampo de la Inteligencia Artificial (IA) que tiene como objetivo la comprensión 
del lenguaje humano por las computadoras. 
Mediante el uso de sus algoritmos es posible el procesamiento masivo de texto para la extracción de información 
relevante de este. Entre las tareas pertenecientes a dicho campo se encuentran Traducción Automática, 
Generación de Lenguaje Natural y Extracción de Argumentos (EA). La EA constituye la identificación y extracción 
automática de las estructuras de inferencia y 
razonamiento expresadas como argumentos presentes en el lenguaje natural [\cite{lawrence2020argument}].
En la actualidad en tareas de PLN como análisis de sentimientos permiten 
extraer cuales son las opiniones o sentimientos presentes, este análisis, sin embargo, presenta una falta 
de información, ya que no presenta el porqué de estas. La EA permite dar respuesta a este problema presentando
los argumentos y cómo sus relaciones justifican sus posiciones. Dicho problema está constituído por diferentes 
estructuras y se compone de distintas tareas necesarias para su solución.

\subsection{Estructuras Argumentativas}

Las estructuras argumentativas son las partes y sus relaciones de las cuales están compuestas la argumentación en los textos.
Estas se componen de dos elementos principales, las Unidades de Discurso Argumentativas (UDA) y los enlaces
existentes entre estas. Las UDAs corresponden a la unidad mínima de argumentación, definida en [\cite{stede2018argumentation}]
como un segmento de texto que juega un solo rol para el argumento siendo analizado, y es 
delimitado por segmentos vecinos que juegan roles diferentes o ningún rol.
Las UDAs se relacionan entre sí formando el proceso de inferencia y razonamiento del argumento.
Tanto los enlaces como las UDAs son clasificadas en dependencia de su rol en la argumentación, estas clasificaciones 
en las UDAs parten de los conceptos de afirmación, declaración controversial y parte central del argumento, y premisa,
razones que la justifican o refutan, y en las relaciones de ataque y apoyo. 

% Esta estructura presentan una representación básica sobre la cual luego se realizan análisis para
% conocer las formas de argumentación empleadas. Un análisis posterior sobre ellas es la extracción
% de esquemas argumentativos, los cuales son patrones de inferencia 
% conectando un conjunto de premisas a una conclusión que representa patrones estereotípicos de 
% razonamiento humano [\cite{lawrence2020argument}], o sea son las maneras en normalmente se realiza 
% la argumentación. Un ejemplo podría ser 

\subsection{Tareas de extracción de argumentos}

Dada las estructuras argumentativas en la EA se conciben las siguientes tareas principales:

\subsubsection{Extracción de UDAs}

Esta tarea constituye en la separación de los segmentos de texto que formarán parte de la estructura.
En este proceso el texto es segmentado y obteniendo un conjunto de UDAs. En el siguiente 
ejemplo\footnote{Traducido del corpus [\cite{stab2017parsing}]} se representa 
la extracción de UDAs marcadas en \emph{cursiva} de un texto dado:

\begin{adjustwidth}{25pt}{25pt}
    En primer lugar, [\emph{el correo electrónico puede contar como uno de los resultados
    más beneficiosos de la tecnología moderna}]. [\emph{Años atrás, las personas pagaban gran cantidad de dinero para 
    enviar sus cartas y sus pagos estaban sujetos al peso de sus cartas o paquetes y muchos accidentes podrían 
    causar problemas que causarían que el correo no fuera enviado}].
\end{adjustwidth}

\subsubsection{Clasificación de UDAs}

La clasificación de UDAs consiste en asignarle la categoría que toma la UDA en la argumentación. En general 
las clasificaciones parten de dos clases bases, las afirmaciones y premisas, aunque estas pueden ser tantas
como sea necesario por el problema en específico a tratar. Siguiendo con el ejemplo, se observa como los 
segmentos extraídos en el paso anterior son clasificados en afirmación y premisa respectivamente: 

\begin{adjustwidth}{25pt}{25pt}
    En primer lugar, [\emph{el correo electrónico puede contar como uno de los resultados
    más beneficiosos de la tecnología moderna}]$_{Afirmación}$. [\emph{Años atrás, las personas pagaban gran cantidad de dinero para 
    enviar sus cartas y sus pagos estaban sujetos al peso de sus cartas o paquetes y muchos accidentes podrían 
    causar problemas que causarían que el correo no fuera enviado}]$_{Premisa}$.
\end{adjustwidth}

\subsubsection{Extracción de relaciones entre las UDAs}

La extracción de relaciones constituye el paso en donde se determina si están relacionadas las UDAs o no. 
La disposición de estas relaciones forma el proceso de razonamiento en que se basa el autor para validar 
su posición. En el ejemplo se representa la existencia de relación mediante su distancia argumentativa con 
la UDA con la que se relaciona. La distancia argumentativa son la cantidad de UDAs del texto que separan la 
UDA fuente del objetivo [\cite{galassi2021deep}], en caso de ser negativa (positiva) el objetivo se encuentra 
antes (después) que la fuente.

\begin{adjustwidth}{25pt}{25pt}
    En primer lugar, [\emph{el correo electrónico puede contar como uno de los resultados
    más beneficiosos de la tecnología moderna}]$_{Afirmación}$. [\emph{Años atrás, las personas pagaban gran cantidad de dinero para 
    enviar sus cartas y sus pagos estaban sujetos al peso de sus cartas o paquetes y muchos accidentes podrían 
    causar problemas que causarían que el correo no fuera enviado}]$_{Premisa, -1}$.
\end{adjustwidth}

\subsubsection{Clasificación de relaciones entre las UDAs}

La clasificación de las relaciones consiste en clasificar las relaciones extraídas en las categorías pertinentes.
Los tipos de relaciones, nacen de dos clases bases por lo general, las relaciones de apoyo y las de ataque.
Las de apoyo se centran en aquellas en las que la UDA fuente afirme la UDA objetivo, las de ataque se basan en 
las que la UDA fuente apoye la negación de la UDA objetivo.

\begin{adjustwidth}{25pt}{25pt}
    En primer lugar, [\emph{el correo electrónico puede contar como uno de los resultados
    más beneficiosos de la tecnología moderna}]$_{Afirmación}$. [\emph{Años atrás, las personas pagaban gran cantidad de dinero para 
    enviar sus cartas y sus pagos estaban sujetos al peso de sus cartas o paquetes y muchos accidentes podrían 
    causar problemas que causarían que el correo no fuera enviado}]$_{Premisa, -1, apoyo}$.
\end{adjustwidth}

Partiendo de esto, se puede observar que las estructuras argumentativas de un texto constituyen un grafo dirigido 
en donde sus nodos representan las UDAs y están anotados con su tipo y sus vértices representan las 
relaciones entre las UDAs y dichos vértices se anotan con el tipo de relación existente entra ambas 
(Figura \ref{fig:arg_struct}).

\begin{figure}[h!]
	\begin{center}
		\begin{center}
            \includesvg[scale=.7]{Graphics/Estructuras_argumentativas.svg}
        \end{center}
	    \caption{Estructuras Argumentativas.}
	\end{center}
\end{figure}\label{fig:arg_struct}

% Argumentation Mining \cite{stede2018argumentation}, capitulo de usos de la argument mining. 