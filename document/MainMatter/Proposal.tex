\chapter{Propuesta}\label{chapter:proposal}

% \begin{itemize}

%     \item Modelo del fenómeno
%     \item Diseño conceptual
%     \begin{itemize}
%         \item Entrada $\rightarrow$ Procesamiento $\rightarrow$ Salida
%         \item Cómo se hace cada etapa
%     \end{itemize}

% \end{itemize}

% Descripción de cada uno de los métodos, modelos, features, medidas de calidad, etc.

La propuesta consiste en un framework para el procesamiento de textos argumentativos entre distintos idiomas.
La entrada de dicho framework es un conjunto de documentos en un lenguaje fuente y su salida es una representación
de las estructuras argumentativas presentes en estos en un lenguaje objetivo. Las estructuras argumentativas 
difieren en dependencia del esquema de anotación usado en los conjuntos de datos usados para concretar el modelo
final. Independiendemente del conjunto de datos el procedimiento se divide en varias partes fundamentales.

\begin{enumerate}
    \item Traducción
    \begin{itemize}
        \item Proyección de corpus (Entrenamiento)
        \item Traducción de documentos (Producción)
    \end{itemize}
    \item Segmentación de UDAs
    \item Extracción de relaciones entre las UDAs
\end{enumerate}

\section{Traducción}

El proceso de traducción consiste en llevar los textos del lenguaje origen al lenguaje final.

\subsection{Traducción de documentos}

La traducción de documentos se realiza automáticamente mediante servicios brindados por Google Traductor.
Este servicio, basado en Traducción Automática Neuronal, posee una gran variedad de lenguajes por lo que
se puede usar en la muchos casos.

\subsection{Proyección}

Para proyectar los conjuntos de datos se establece una serie de tareas:

\begin{enumerate}
    \item Creación de oraciones alineadas en lenguaje fuente y lenguaje objetivo.
    \begin{enumerate}
        \item Segmentar oraciones del texto.
        \item Traducir oraciones del lenguaje fuente al lenguaje objetivo.
    \end{enumerate}
    \item Creación de alineaciones de palabra entre las oraciones alineadas.
    \item Proyección de las etiquetas del lenguaje fuente al lenguaje objetivo. 
\end{enumerate}

Los textos son separados en oraciones debido a que estas poseen suficiente carga semántica
para poder ser traducida de manera independiente sin perder mucho sentido al ser concatenadas
luego. El orden de las oraciones tiene que ser preservado para no alterar las relaciones entre 
las UDA en los datos. 

Las alineaciones de las palabras son calculadas por cada par de oraciones mediante el uso de 
herramientas como FastAlign o AwesomeAlign. Luego estos datos son usados en la proyección de 
las etiquetas para la conformación del corpus final en el lenguaje objetivo.

\section{Segmentación}

La segmentación es el proceso de extraer del texto plano las UDAs. Para esto se propone un modelo
tomando como base los trabajos de [\cite{ma2016end}]. Se modela el problema como un problema de anotación
secuencia a secuencia en donde las etiquetas BIOES son usadas para delimitar los límites de las 
UDAs, la B indica inicio de UDA, I indica el interior de una UDA, O indica que el token no pertence
a una UDA, E indica el final de UDA y S es una anotación que indica una UDA de un solo token. 
Este esquema de anotación de secuencias es más complejo que BIO, aunque es más indicado
para versiones bidireccionales de secuencias ya que hace explícitas las delimitaciones de las 
secuencias.

Los features usados para los tokens de la secuencia entrada del modelo son los embeddings
GloVe [\cite{pennington2014glove}]. Opcionalmente se usan features aprendidos de la morfología
del propio token mediante Redes Convolucionales o Long-Short Term Memory aplicados a embeddings
de caracteres. El modelo posee la habilidad de agregar las etiquetas Part-of-Speech, aunque este
atributo requiere de un método de anotación de estas por lo cual le resta libertad para
elegir lenguajes fuentes y objetivos.

El modelo consiste en la codificación de la secuencia en los atributos previamente descritos,
luego es aplicado una Long-Short Term Memory sobre la secuencia resultante. Conexiones residuales, 
normalizaciones y dropouts son agregadas entre capas. En la capa final se presenta un Conditional
Random Field encargado de la conversión final de los features de las secuencias en etiquetas BIOES.

Es posible realizar, además de la segmentación, la clasificación de las UDA mediante el agrego de
información a las etiquetas BIOES del tipo de UDA que es.

Entre las medidas usadas para observar el desempeño de la segmentación se encuentra F1-Macro.
Esta es comúnmente usada en problemas de clasificación múltiples para y es usada como referencia
en la mayoría de las investigaciones estudiadas. Otra medida es \%F1, la definición de esta medida
es similar a la de F1 pero sus elementos son tratados como coincidencias pariales de segmentos en vez
de etiquetas asignadas a tokens, se considera que las secuencias coinciden si al menos se intersectan 
en un porciento mayor o igual al establecido en la medida, por ejemplo, en caso de 100\%F1 se mide
las coincidencias exactas de las secuencias. Esta medida es calculada también al 50\%, la razón es
permitir cierta libertad en el comienzo y final de las UDAs. 

\section{Extracción de relaciones}

La extracción de relaciones constituye en el proceso de extraer las conexiones existentes
entre las UDAs y clasificar estas. El problema es modelado como dado un par de UDAs, una origen
y otra objetivo, predecir
la relación entre estas, para esto se agrega una nueva categoría de relaciones del esquema de
anotación que indica la no relación de UDAs. El modelo realiza conjuntamente la clasificación
de los tipos de UDA como tarea auxiliar.

Los features usados son las representaciones GloVe de las palabras de las UDAs entradas. 
Estos son codificados mediante una Long-Short Term Memory. Los features de ambas partes
fuente y destino son pasados por un mecanismo de atención y una red residual para finalmente
ser la entrada a tres clasificadores los cuales dictan el tipo de relación entre estos y el tipo
de UDA que son cada una de las dos partes.

Para la evaluación del modelo se utilizó la medida F1-Macro por cada clasificador.